\documentclass[a4paper]{article}

% Some basic packages
\pdfminorversion=7

\usepackage[utf8]{inputenc}
\usepackage[T1]{fontenc}
\usepackage{textcomp}
\usepackage[english]{babel}
\usepackage{url}
\usepackage{graphicx}
\usepackage{float}
\usepackage{booktabs}
\usepackage{enumitem}

% Don't indent paragraphs, leave some space between them
\usepackage{parskip}

% Hide page number when page is empty
\usepackage{emptypage}
\usepackage{subcaption}
\usepackage{multicol}
\usepackage{xcolor}

% Other font I sometimes use.
% \usepackage{cmbright}

% Math stuff
\usepackage{amsmath, amsfonts, mathtools, amsthm, amssymb}
% Fancy script capitals
\usepackage{mathrsfs}
\usepackage{cancel}
% Bold math
\usepackage{bm}
% Some shortcuts
\newcommand\N{\ensuremath{\mathbb{N}}}
\newcommand\R{\ensuremath{\mathbb{R}}}
\newcommand\Z{\ensuremath{\mathbb{Z}}}
\renewcommand\O{\ensuremath{\emptyset}}
\newcommand\Q{\ensuremath{\mathbb{Q}}}
\newcommand\C{\ensuremath{\mathbb{C}}}
\newcommand\mathP{\ensuremath{\mathbb{P}}}
\newcommand{\tens}[1]{%
	\mathbin{\mathop{\otimes}\limits_{#1}}%
}
\DeclareMathOperator{\Aut}{Aut}
\DeclareMathOperator{\Inn}{Inn}
\DeclareMathOperator{\Hom}{Hom}
\DeclareMathOperator{\Tr}{Tr}
% Easily typeset systems of equations (French package)
\usepackage{systeme}

% Put x \to \infty below \lim
\let\svlim\lim\def\lim{\svlim\limits}

%Make implies and impliedby shorter
\let\implies\Rightarrow
\let\impliedby\Leftarrow
\let\iff\Leftrightarrow
\let\epsilon\varepsilon

% Add \contra symbol to denote contradiction
\usepackage{stmaryrd} % for \lightning
\newcommand\contra{\scalebox{1.5}{$\lightning$}}

% \let\phi\varphi

% Command for short corrections
% Usage: 1+1=\correct{3}{2}

\definecolor{correct}{HTML}{009900}
\newcommand\correct[2]{\ensuremath{\:}{\color{red}{#1}}\ensuremath{\to }{\color{correct}{#2}}\ensuremath{\:}}
\newcommand\green[1]{{\color{correct}{#1}}}

% horizontal rule
\newcommand\hr{
    \noindent\rule[0.5ex]{\linewidth}{0.5pt}
}

% hide parts
\newcommand\hide[1]{}

% si unitx
\usepackage{siunitx}
\sisetup{locale = FR}

% Environments
\makeatother
% For box around Definition, Theorem, \ldots
\usepackage{mdframed}
\mdfsetup{skipabove=1em,skipbelow=0em}
\theoremstyle{definition}
\newmdtheoremenv[nobreak=true]{definitie}{Definitie}
\newmdtheoremenv[nobreak=true]{eigenschap}{Eigenschap}
\newmdtheoremenv[nobreak=true]{gevolg}{Gevolg}
\newmdtheoremenv[nobreak=true]{lemma}{Lemma}
\newmdtheoremenv[nobreak=true]{propositie}{Propositie}
\newmdtheoremenv[nobreak=true]{definition}{Definition}
\newtheorem*{eg}{Example}
\newtheorem*{notation}{Notation}
\newtheorem*{previouslyseen}{As previously seen}
\newtheorem*{remark}{Remark}
\newtheorem*{note}{Note}
\newtheorem*{problem}{Problem}
\newtheorem*{observe}{Observe}
\newtheorem*{property}{Property}
\newtheorem*{intuition}{Intuition}
\newmdtheoremenv[nobreak=true]{prop}{Proposition}
\newmdtheoremenv[nobreak=true]{theorem}{Theorem}
\newmdtheoremenv[nobreak=true]{corollary}{Corollary}

% End example and intermezzo environments with a small diamond (just like proof
% environments end with a small square)
\usepackage{etoolbox}
\AtEndEnvironment{vb}{\null\hfill$\diamond$}%
\AtEndEnvironment{intermezzo}{\null\hfill$\diamond$}%
% \AtEndEnvironment{opmerking}{\null\hfill$\diamond$}%

% Fix some spacing
% http://tex.stackexchange.com/questions/22119/how-can-i-change-the-spacing-before-theorems-with-amsthm
\makeatletter
\def\thm@space@setup{%
  \thm@preskip=\parskip \thm@postskip=0pt
}


% \lecture starts a new lecture (les in dutch)
%
% Usage:
% \lecture{1}{di 12 feb 2019 16:00}{Inleiding}
%
% This adds a section heading with the number / title of the lecture and a
% margin paragraph with the date.

% I use \dateparts here to hide the year (2019). This way, I can easily parse
% the date of each lecture unambiguously while still having a human-friendly
% short format printed to the pdf.

\usepackage{xifthen}
\def\testdateparts#1{\dateparts#1\relax}
\def\dateparts#1 #2 #3 #4 #5\relax{
    \marginpar{\small\textsf{\mbox{#1 #2 #3 #5}}}
}

\def\@lecture{}%
\newcommand{\lecture}[3]{
    \ifthenelse{\isempty{#3}}{%
        \def\@lecture{Lecture #1}%
    }{%
        \def\@lecture{Lecture #1: #3}%
    }%
    \subsection*{\@lecture}
    \marginpar{\small\textsf{\mbox{#2}}}
}

\DeclareMathOperator{\Ima}{Im}

% These are the fancy headers
\usepackage{fancyhdr}
\pagestyle{fancy}

% LE: left even
% RO: right odd
% CE, CO: center even, center odd
% My name for when I print my lecture notes to use for an open book exam.
% \fancyhead[LE,RO]{Gilles Castel}

\fancyhead[RO,LE]{\@lecture} % Right odd,  Left even
\fancyhead[RE,LO]{}          % Right even, Left odd

\fancyfoot[RO,LE]{\thepage}  % Right odd,  Left even
\fancyfoot[RE,LO]{}          % Right even, Left odd
\fancyfoot[C]{\leftmark}     % Center

\makeatother




% Todonotes and inline notes in fancy boxes
\usepackage{todonotes}
\usepackage{tcolorbox}

% Make boxes breakable
\tcbuselibrary{breakable}

% Figure support as explained in my blog post.
\usepackage{import}
\usepackage{xifthen}
\usepackage{pdfpages}
\usepackage{transparent}
\newcommand{\incfig}[1]{%
    \def\svgwidth{\columnwidth}
    \import{./figures/}{#1.pdf_tex}
}

% Fix some stuff
% %http://tex.stackexchange.com/questions/76273/multiple-pdfs-with-page-group-included-in-a-single-page-warning
\pdfsuppresswarningpagegroup=1


% My name
\author{Jakub Wornbard}


\usepackage[utf8]{inputenc}
\usepackage[T1]{fontenc}
\usepackage{textcomp}
\usepackage[english]{babel}
\usepackage{amsmath, amssymb}
\newtheorem{thm}{Theorem}
\newtheorem{lem}[thm]{Lemma}
\newtheorem{exmp}[thm]{Example}
\newtheorem{defn}[thm]{Definition}
\newtheorem{que}[thm]{Question}
\newtheorem{clm}[thm]{Claim}
\pdfsuppresswarningpagegroup=1

\begin{document}
\section{Upper bounds on $\Gamma$ for various manifolds}
\begin{enumerate}
    \item $\C\mathP^{n}$.\\ The corresponding potential is $W=\sum_{i=1}^{n} x_{i} + \frac{1}{\prod_{j=1}^{n} x_{j} }$. The fact that $\Gamma$ has to act on $x_{i}$ in a way that permutes the terms of $W$ implies that it has to be a subgroup of $S_{n+1}$.\\
        The partial derivatives of $W$ are $\frac{\partial w}{\partial x_{i}} =1-\frac{1}{x_{i}\prod_{j=1}^{n}x_{j}}$. Therefore the critical points of $W$ are those where all  $x_{i}$ are all equal. For even $n$ they can only be equal to $1$. For odd $n$ it's $\pm 1$\\
        Thus the permutations of terms of $W$ also preserve the fixed points. Hence this doesn't impose any further restrictions of $\Gamma$.\\
        \textbf{TODO: Write something about the realization of these on the Clifford algebra and also the functions that actually give rise to $\Gamma$ i.e. the $U(n)$ stuff}.\\
    \item Blowup of $\C\mathP^{n}$ along a $k$-dimensional face.\\
        WLOG $W=\sum x_{i}+\frac{1}{\prod ^n x_{i}}+\frac{1}{\prod ^k x_{i}}$ where the last term is the normal vector to the new facet obtained by the blowup.\\
        Consider the corresponding critical points. For $j\le k$ we have $\frac{\partial W}{\partial x_{j}} =1-\frac{1}{x_{j}\prod^n x_{i}} - \frac{1}{x_{j}\prod^k x_{i}}$. For greater $j$ we only have the first term.
        Since for fixed values of these products of $x_{i}$ there exist unique solutions $x_{j}$to $\partial_j W=0$, it must be the case that at critical points for $i\le k$ all $x_{i}$     are equal. The same holds for $i>k$ for the same reason. \\
        The solutions to these two equations are distinct becasue if the gradient is 0 at coordinates $>  k$ then if for smaller coordinates we had the same value of  $x_{i}$ we would require $\frac{1}{x^{k+1}}=0$ which is nonsense. Therefore  no element of $\Gamma $ permutes the basis vector in a way which maps $x_{i\le k}$ to $x_{j>k}$.\\
        None of the $x_{i\le k}$ can be mapped to $\frac{1}{\prod^n x_{i}}$ either because then at critical points these two terms would have to be equal but then by definition of the critical points we would have $\frac{1}{\prod^{k}x_{j}}=0$ which is nonsense.\\
        For the same reason $\Gamma$ can't send $x_{i\le k}$ to $\frac{1}{\prod^{k}x_{j}}$.\\
        However, permuting the $x_{i\le k}$ among themselves is definitely possible. We can also permute $x_{i>k}$ and $\frac{1}{\prod^{n}x_{i}}    $ among themselves. Indeed it is possible combinatorially i.e. preserves the set of basis vectors. It remains to verify that it also pre    serves fixed points.
        That is we want to have  $x_{i>k}=\frac{1}{\prod^{n}x_{i}}$  at the critical points. This is precisely the condition for the  >k  coordinates of the gradient to be zero, so such permutations are not excluded by our conditions.\\
       In conclusion this (without accounting for the Clifford algebras, which is not necessary at least if the critical points are non-degenerate) gives us  $\Gamma \le S_{k} \times  S_{n-k+1}$.
   \item Square-based pyramid \\
       Consider a pyramid with base sides parallel to the $x,y$ axes. Then $W=z+\frac{x}{z}+\frac{y}{z}+\frac{1}{xz}+\frac{1}{yz}$. 
       $D_8$ generated by a rotation  $\begin{pmatrix} 0 & -1 &0\\ 1 & 0 & 0\\ 0& 0 &1 \end{pmatrix} $ and reflection $\begin{pmatrix} -1 & 0 & 0 \\ 0 & -1 & 0\\ 0  & 0 &1 \end{pmatrix} $ clearly fix $W$.\\
       The critical points of $W$ are $\left( -1,-1,2i \right) ,\left( -1,-1,2i \right), \left( 1,1,-    2 \right),\left( 1,1,2 \right) $. It is a straightforward computation to verify that these are fixed by our $D_8$.\\
       Now we have to show that larger subgroups of  $S_5$ are excluded by the same conditions that $D_8$ satisfies. Sending $z$ to any of the other terms in $W$ clearly doesn't preserve the critical points, so it has to be fixed by all elements of $\Gamma$.\\
       This reduces $\Gamma$ to at most a subgroup of $S_4$. Since $z$ is fixed, the possible images of $x,y$ are $x,y,\frac{1}{x},\frac{1}{y}$. However, note that if $x\mapsto y$ then we can't have $y\mapsto \frac{1}{y}$ because that implies $\frac{1}{y}\mapsto y$ and have $x$ can't be mapped to $ y$. In the same way we exclude the analogous mapping with $x,y$ swapped. Thus the number of possible images of $x,y$ is $2\cdot 2 +2\cdot 2$ corresponding to sending  $x$ to $x,\frac{1}{x}$ or $y,\frac{1}{y}$. Thus $\Gamma$ has at most 8 elements. Hence $D_8$ is the best possible upper bound based on the condition of fixing $W$ and its critical points.
   \item $\C\mathP^{n_1}\times \ldots \C\mathP^{n_{k}}$.\\
       For $\C\mathP^{n}$ one class of critical points is those where all arguments of the potential are equal to the same $n+1$-st (or th?) root of unity (not necessarily a primitive one). This is apparent from the form of $W(x_1, \ldots, x_n)=\sum_{i=1}^{n}x_{i}+\frac{1}{\prod_{i=1}^{n}x_{i}}$. Now consider a direct product of such spaces. It's potential is the sum of potentials of the separate spaces i.e. $W=W\left( \underline{x_1}, \ldots, \underline{x_{k}} \right) =\sum_{i=1}^{k} W_i(\underline{x_{i}})$ with $\underline{x_i}=\left( x_{i,1}, \ldots , x_{i,n_{i}} \right) $ and $W_i$ the potential of $\C\mathP^{n_{i}}$.\\
       Hence critical points of $W$ are the points for which $\underline{x_{i}}$ is the critical point of $W_{i}$. This tells us two things. Firstly if an element of $\Gamma$ only permutes the terms of $W$ within each $W_i$ then it definitely preserves the fixed points because we know that all permutations within  $\C\mathP^{n_i}$ are possible.\\
       Secondly no permutation can move terms between $\underline{x_{k}}$ and $ \underline{x_{j}}$ with $k\neq j$. That is because it would not fix a critical point where $\underline{x_k}=\left( 1, \ldots, 1 \right) $ and $x_{j}=\left( e^{\frac{i\pi}{n_j+1}}, \ldots, e^{\frac{i\pi}{n_j+1}}  \right) $. Therefore the $\Gamma$ of the prouct of $\C\mathP$ 's is a subgroup of the product of their respective Lagrangian monodromy groups.\\
   \item Blowup of $\C\mathP^{2n}$ at a vertex.\\
       Wlog the potential is $W=\sum_{i=1}^{2n}\left( x_{i} +\frac{1}{x_{i}}\right) +\frac{1}{\prod_{i=1}^{2n}x_{i}}$.\\
       Then $\frac{\partial W}{\partial x_{i}} =1-\frac{1}{x_{i}^2}-\frac{1}{x_{i}\prod_{j=1}^{2n}x_{j}}$. Thus setting $x_{i}=\sqrt{2} $ and for $i\neq j$ $x_{j}=1$ gives us a critical point. A general element of $\Gamma$ can map $x_{i}$ to one of $x_{j}$,$\frac{1}{x_{j}}$ or $\frac{1}{\prod x_{j}}$. But at that critical point, only mapping $x_{i}$ to itself preserves the fixed point. Thus each $x_{i}$ is fixed by all elements of $\Gamma$ so the group is trivial.
\end{enumerate} 
	\section{Important fact}
	Let $x = (x_1, x_2, ..., x_n) \in (C^{\star})^n$
	
	Write $\textbf{x}^v$ for $x_1^{v_1} x_2^{v_2} .. x_n^{v_n}$. Then, if a matrix $M$ fixes $\textbf{x}$, then we have $\textbf{x}^v = \textbf{x}^{Mv}$ for any vector $v$.
	Consider a term 
	$$f(\textbf{x}) = \sum_{i=0}^{k-1} \textbf{x}^{M^i v}$$
	where $M^k v =v$.
	Then we have(assuming $M$ fixes $x$):
	$$(\nabla f)_j = \sum_{i=0}^{k-1} (M^i v)_j \textbf{x}^{M^i v} = \sum_{i=0}^{k-1} (M^i v)_j \textbf{x}^{v} = (\sum_{i=0}^{k-1} M^i) \textbf{x}^{v} v_j$$
	So $$\nabla f = (\sum_{i=0}^{k-1} M^i) \textbf{x}^{v} v$$.
	$$(M-I)\nabla f= (M^k-I)v \textbf{x}^{v}= 0$$
	Now $\Gamma$ fixes $W$, so we can partition $W$ into linear combination of terms like $f$, so:
	$$(M-I)\nabla W = 0$$
    \section{Classification in dimension 2}
    	Let $M \subset Z^2$ be the set of exponents of monomials in $W(x,y)$.
    
    \clm If $M$ contains two linearly independent vectors, then every element in $\Gamma$ has finite order.
    Suppose that $M$ contains two linearly independent vectors $x$ and $y$. Then for any $A \in \Gamma$ we have $A^k x= x$ for some $k$ (because $M$ is finite). Similarly $A^m y = y$ for some $m$. Hence $A^{k m}$ fixes both $x$ and $y$, so $A^{k m} = I$, so $A$ has finite order.
    \clm If $M$ contains no two independent vectors, then $\Gamma_+ \cong \mathbb{Z}$ or $\Gamma_+ \cong \mathbb{Z} \times C_{2}$ or $\Gamma_+ \cong C_{2}$.
    
    Let $(a,b) \in M$ be the vector with minimal length.
    Choose a basis for $Z^2$ such that $(a,b)$ is a multiple of $(1,0)$ in a new basis. Then $(1,0)$ is an eigenvector of $A$ for any $A \in \Gamma_+$. Moreover, the eigenvalue is either $1$ or $-1$, because otherwise we could obtain an infinite sequence of distinct monomials. For any $A \in \Gamma_+$ we also need the determinant to be 1, so it either has the form:
    $\begin{pmatrix}
    1 & a \\
    0 & 1 \\
    \end{pmatrix}$ or the form 
    $\begin{pmatrix}
    -1 & b \\
    0 & -1 \\
    \end{pmatrix}$
    for some $a, b$ in $\mathbb{Z}$.
    
    $\begin{pmatrix}
    1 & 1 \\ 
    0 & 1 \\
    \end{pmatrix}$ or the form 
    $\begin{pmatrix}
    -1 & 1 \\
    0 & -1 \\
    \end{pmatrix}$ generate a group isomorphic to $\mathbb{Z} \times C_2$, and $\Gamma_+$ is its subgroup so $\Gamma_+ \cong \mathbb{Z}$ or $\Gamma_+ \cong \mathbb{Z} \times C_2$ or $\Gamma_+ \cong C_2$
    \clm If $M$ contains two linearly independent vectors, then $\Gamma_+ \cong C_2$ or $\Gamma_+ \cong C_3$.
    
    Suppose that $\Gamma_+$ contains an element of order 2. It has determinant $1$, so it either has eigenvalues $1, 1$ or $-1, -1$. Direct check of conjugacy classes shows that it has to be conjugate to $\begin{pmatrix}
    -1 & 0 \\ 
    0 & -1 \\
    \end{pmatrix}$, hence equal to it. We know that then $\Gamma_+$ has no other elements of finite order, so no other elements at all, hence $\Gamma_+ \equiv C_2$.
    The only other possibility is that $\Gamma_+$ has exponent 3. Then by considering 3rd roots of unity we see that any non-identity element $A$ in $\Gamma_+$ has eigenvalues $\frac{-1-i\sqrt{3}}{2}$ and $\frac{-1+i\sqrt{3}}{2}$. Hence, we can pick an appropriate basis and assume that $\begin{pmatrix}
    \frac{-1-i\sqrt{3}}{2} & 0 \\ 
    0 &  \frac{-1+i\sqrt{3}}{2}\\
    \end{pmatrix} \in \Gamma_+$.In particular, any non-identity element of $A$ has trace $-1$ and determinant $1$. Suppose $\Gamma_+$ contains a matrix with non-zero top right entry. It then has the form:
    $\begin{pmatrix}
    a & b \\ 
    \frac{-1-a-a^2}{b} & -1-a \\
    \end{pmatrix}$ 
    We need 
    $$\begin{pmatrix}
    \frac{-1-i\sqrt{3}}{2} & 0 \\ 
    0 &  \frac{-1+i\sqrt{3}}{2}\\
    \end{pmatrix} \begin{pmatrix}
    a & b \\ 
    \frac{-1-a-a^2}{b} & -1-a \\
    \end{pmatrix}$$ to have trace $-1$. Hence $a = \frac{-1-i\sqrt{3}}{2}$, and so left down entry is $0$. 
    But then 
    $$\begin{pmatrix}
    \frac{-1-i\sqrt{3}}{2} & 0 \\ 
    0 &  \frac{-1+i\sqrt{3}}{2}\\
    \end{pmatrix} \begin{pmatrix}
    a & b \\ 
    \frac{-1-a-a^2}{b} & -1-a \\
    \end{pmatrix} = \begin{pmatrix}
    \frac{-1+i\sqrt{3}}{2} & \frac{-b-b i\sqrt{3}}{2} \\ 
    0 & \frac{-1-i\sqrt{3}}{2} \\
    \end{pmatrix} \in \Gamma_+$$
    But it doesn't satisfy these conditions.
    Similarly matrix with non-zero left down entry cannot exist, so all elements of $\Gamma_+$ are diagonal, and $\Gamma_+ \cong C_3$
    \clm If $C_6 \in \Gamma$, then $C_6\in \Gamma_+$\\
    Suppose we had an element of order 6 in $GL(2,\Z)\setminus SL(2,\Z)$. Consider its eigenvalues $\lambda_1,\lambda_2$. Let them have multiplicative orders $a,b$ s.t. $LCM(a,b)=6$. The possibilities are (1,6),(2,6),(3,6),(6,6),(2,3) corresponding to different choices of roots of unity for  $\lambda_1.\lambda_2$. But of these only the (6,6) pair can add up to an integer since sums of roots of unity of different order can't be integers (the sum of phases has to be 0). Now, a product of a root of unity and its conjugate has to be 1, hence the determinant is not -1, contradiction.\\
    
    Summary of nontrivial possibilities for group $\Gamma$:
    \begin{enumerate}
    	\item $C_2$: Given by $W(x,y) = x^2 + y^2+ x y$ 
    	\item $C_2 \times C_2$: Given by $W(x,y) = x+ y +\frac{1}{x}+\frac{1}{y}$
    	\item $C_3$: We haven't found a corresponding $W(x,y)$ yet
    	\item $C_4$: We haven't found a corresponding $W(x,y)$ yet
    	\item $S_3$: Given by $W(x,y) = x + y +\frac{1}{xy}$
    	\item $C_6$: We haven't found a corresponding $W(x,y)$ yet
    	\item $\mathbb{Z}$: We haven't found a corresponding $W(x,y)$ yet
    	\item $\mathbb{Z} \times C_2$: We haven't found a corresponding $W(x,y)$ yet
    	\item $\mathbb{Z} \times C_4$: We haven't found a corresponding $W(x,y)$ yet
    	\item $\mathbb{Z} \times C_2 \times C_2$: We haven't found a corresponding $W(x,y)$ yet
    	\item $D_\infty$: $W(x,y) = x^n$(not sure about this one)
    	\item $D_\infty \times C_2$: We haven't found a corresponding $W(x,y)$ yet
    	\item Possibly something else with index 2 subgroup isomorphic to $\mathbb{Z} \times C_2$(We're not sure if other groups exist)
    \end{enumerate}
	\section{Conjecture about groups $\Gamma$ for $n \geq 5$}
		In dimension $n$, $\Gamma$ is isomorphic to a subgroup of a group of the form:
		$$S_{a_1} \times S_{a_2} \times \dots S_{a_k}$$ for some $k, a_1, a_2, \dots a_k$ with $\sum_{i=1}^k (a_i - 1) = n$.

\end{document}
