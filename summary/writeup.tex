\documentclass[a4paper]{article}

\input{preamble.tex}
\usepackage[utf8]{inputenc}
\usepackage[T1]{fontenc}
\usepackage{textcomp}
\usepackage[english]{babel}
\usepackage{amsmath, amssymb}
\newtheorem{thm}{Theorem}
\newtheorem{lem}[thm]{Lemma}
\newtheorem{exmp}[thm]{Example}
\newtheorem{defn}[thm]{Definition}
\newtheorem{que}[thm]{Question}
\pdfsuppresswarningpagegroup=1

\begin{document}
Upper bounds on $\Gamma$ for various manifolds
\begin{enumerate}
    \item $\C\mathP^{n}$.\\ The corresponding potential is $W=\sum_{i=1}^{n} x_{i} + \frac{1}{\prod_{j=1}^{n} x_{j} }$. The fact that $\Gamma$ has to act on $x_{i}$ in a way that permutes the terms of $W$ implies that it has to be a subgroup of $S_{n+1}$.\\
        The partial derivatives of $W$ are $\frac{\partial w}{\partial x_{i}} =1-\frac{1}{x_{i}\prod_{j=1}^{n}x_{j}}$. Therefore the critical points of $W$ are those where all  $x_{i}$ are all equal. For even $n$ they can only be equal to $1$. For odd $n$ it's $\pm 1$\\
        Thus the permutations of terms of $W$ also preserve the fixed points. Hence this doesn't impose any further restrictions of $\Gamma$.\\
        \textbf{TODO: Write something about the realization of these on the Clifford algebra and also the functions that actually give rise to $\Gamma$ i.e. the $U(n)$ stuff}.\\
    \item Blowup of $\C\mathP^{n}$ along a $k$-dimensional face.\\
        WLOG $W=\sum x_{i}+\frac{1}{\prod ^n x_{i}}+\frac{1}{\prod ^k x_{i}}$ where the last term is the normal vector to the new facet obtained by the blowup.\\
        Consider the corresponding critical points. For $j\le k$ we have $\frac{\partial W}{\partial x_{j}} =1-\frac{1}{x_{j}\prod^n x_{i}} - \frac{1}{x_{j}\prod^k x_{i}}$. For greater $j$ we only have the first term.
        Since for fixed values of these products of $x_{i}$ there exist unique solutions $x_{j}$to $\partial_j W=0$, it must be the case that at critical points for $i\le k$ all $x_{i}$     are equal. The same holds for $i>k$ for the same reason. \\
        The solutions to these two equations are distinct becasue if the gradient is 0 at coordinates $>  k$ then if for smaller coordinates we had the same value of  $x_{i}$ we would require $\frac{1}{x^{k+1}}=0$ which is nonsense. Therefore  no element of $\Gamma $ permutes the basis vector in a way which maps $x_{i\le k}$ to $x_{j>k}$.\\
        None of the $x_{i\le k}$ can be mapped to $\frac{1}{\prod^n x_{i}}$ either because then at critical points these two terms would have to be equal but then by definition of the critical points we would have $\frac{1}{\prod^{k}x_{j}}=0$ which is nonsense.\\
        For the same reason $\Gamma$ can't send $x_{i\le k}$ to $\frac{1}{\prod^{k}x_{j}}$.\\
        However, permuting the $x_{i\le k}$ among themselves is definitely possible. We can also permute $x_{i>k}$ and $\frac{1}{\prod^{n}x_{i}}    $ among themselves. Indeed it is possible combinatorially i.e. preserves the set of basis vectors. It remains to verify that it also pre    serves fixed points.
        That is we want to have  $x_{i>k}=\frac{1}{\prod^{n}x_{i}}$  at the critical points. This is precisely the condition for the  >k  coordinates of the gradient to be zero, so such permutations are not excluded by our conditions.\\
       In conclusion this (without accounting for the Clifford algebras, which is not necessary at least if the critical points are non-degenerate) gives us  $\Gamma \le S_{k} \times  S_{n-k+1}$.
   \item Square-based pyramid \\
       Consider a pyramid with base sides parallel to the $x,y$ axes. Then $W=z+\frac{x}{z}+\frac{y}{z}+\frac{1}{xz}+\frac{1}{yz}$. 
       $D_8$ generated by a rotation  $\begin{pmatrix} 0 & -1 &0\\ 1 & 0 & 0\\ 0& 0 &1 \end{pmatrix} $ and reflection $\begin{pmatrix} -1 & 0 & 0 \\ 0 & -1 & 0\\ 0  & 0 &1 \end{pmatrix} $ clearly fix $W$.\\
       The critical points of $W$ are $\left( -1,-1,2i \right) ,\left( -1,-1,2i \right), \left( 1,1,-    2 \right),\left( 1,1,2 \right) $. It is a straightforward computation to verify that these are fixed by our $D_8$.\\
       Now we have to show that larger subgroups of  $S_5$ are excluded by the same conditions that $D_8$ satisfies. Sending $z$ to any of the other terms in $W$ clearly doesn't preserve the critical points, so it has to be fixed by all elements of $\Gamma$.\\
       This reduces $\Gamma$ to at most a subgroup of $S_4$. Since $z$ is fixed, the possible images of $x,y$ are $x,y,\frac{1}{x},\frac{1}{y}$. However, note that if $x\mapsto y$ then we can't have $y\mapsto \frac{1}{y}$ because that implies $\frac{1}{y}\mapsto y$ and have $x$ can't be mapped to $ y$. In the same way we exclude the analogous mapping with $x,y$ swapped. Thus the number of possible images of $x,y$ is $2\cdot 2 +2\cdot 2$ corresponding to sending  $x$ to $x,\frac{1}{x}$ or $y,\frac{1}{y}$. Thus $\Gamma$ has at most 8 elements. Hence $D_8$ is the best possible upper bound based on the condition of fixing $W$ and its critical points.

\end{enumerate} 
\end{document}
