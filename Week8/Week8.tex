\documentclass[a4paper]{article}

\input{preamble.tex}
\usepackage[utf8]{inputenc}
\usepackage[T1]{fontenc}
\usepackage{textcomp}
\usepackage[english]{babel}
\usepackage{amsmath, amssymb}
\newtheorem{thm}{Theorem}
\newtheorem{lem}[thm]{Lemma}
\newtheorem{exmp}[thm]{Example}
\newtheorem{defn}[thm]{Definition}
\newtheorem{que}[thm]{Question}
\newtheorem{clm}[thm]{Claim}
\pdfsuppresswarningpagegroup=1

\begin{document}
	Let $M \subset Z^2$ be the set of exponents of monomials in $W(x,y)$.
	
\clm If $M$ contains two linearly independent vectors, then every element in $\Gamma$ has finite order.

 Suppose that $M$ contains two linearly independent vectors $x$ and $y$. Then for any $A \in \Gamma$ we have $A^k x= x$ for some $k$ (because $M$ is finite). Similarly $A^m y = y$ for some $y$. Hence $A^{k m}$ fixes both $x$ and $y$, so $A^{k m} = I$, so $A$ has finite order.
\clm If $M$ contains no two independent vectors, then $\Gamma_+ \cong \mathbb{Z}$ or $\Gamma_+ \cong \mathbb{Z} \times C_{2}$ or $\Gamma_+ \cong C_{2}$.

Let $(a,b) \in M$ be the vector with minimal length.
Choose a basis for $Z^2$ such that $(a,b)$ is a multiple of $(1,0)$ in a new basis. Then $(1,0)$ is an eigenvector of $A$ for any $A \in \Gamma_+$. Moreover, the eigenvalue is either $1$ or $-1$, because otherwise we could obtain an infinite sequence of distinct monomials. For any $A \in \Gamma_+$ we also need the determinant to be 1, so it either has the form:
$\begin{pmatrix}
	1 & a \\
	0 & 1 \\
\end{pmatrix}$ or the form 
$\begin{pmatrix}
-1 & b \\
0 & -1 \\
\end{pmatrix}$
for some $a, b$ in $\mathbb{Z}$.

$\begin{pmatrix}
1 & 1 \\ 
0 & 1 \\
\end{pmatrix}$ or the form 
$\begin{pmatrix}
-1 & 1 \\
0 & -1 \\
\end{pmatrix}$ generate a group isomorphic to $\mathbb{Z} \times C_2$, and $\Gamma_+$ is its subgroup so $\Gamma_+ \cong \mathbb{Z}$ or $\Gamma_+ \cong \mathbb{Z} \times C_2$ or $\Gamma_+ \cong C_2$
\clm If $M$ contains two linearly independent vectors, then $\Gamma_+ \cong C_2$ or $\Gamma_+ \cong C_3$
Suppose that $\Gamma_+$ contains an element of order 2. It has determinant $1$, so it either has eigenvalues $1, 1$ or $-1, -1$. Direct check of conjugacy classes shows that it has to be conjugate to $\begin{pmatrix}
-1 & 0 \\ 
0 & -1 \\
\end{pmatrix}$, hence equal to it. We know that then $\Gamma_+$ has no other elements of finite order, so no other elements at all, hence $\Gamma_+ \equiv C_2$.
The only other possibility is that $\Gamma_+$ has exponent 3. Then by considering 3rd roots of unity we see that any non-identity element $A$ in $\Gamma_+$ has eigenvalues $\frac{-1-i\sqrt{3}}{2}$ and $\frac{-1+i\sqrt{3}}{2}$. Hence, we can pick an appropriate basis and assume that $\begin{pmatrix}
\frac{-1-i\sqrt{3}}{2} & 0 \\ 
0 &  \frac{-1+i\sqrt{3}}{2}\\
\end{pmatrix} \in \Gamma_+$.In particular, any non-identity element of $A$ has trace $-1$ and determinant $1$. Suppose $\Gamma_+$ contains a matrix with non-zero top right entry. It then has the form:
$\begin{pmatrix}
a & b \\ 
\frac{-1-a-a^2}{b} & -1-a \\
\end{pmatrix}$ 
We need 
$$\begin{pmatrix}
\frac{-1-i\sqrt{3}}{2} & 0 \\ 
0 &  \frac{-1+i\sqrt{3}}{2}\\
\end{pmatrix} \begin{pmatrix}
	a & b \\ 
	\frac{-1-a-a^2}{b} & -1-a \\
\end{pmatrix}$$ to have trace $-1$. Hence $a = \frac{-1-i\sqrt{3}}{2}$, and so left down entry is $0$. 
But then 
$$\begin{pmatrix}
\frac{-1-i\sqrt{3}}{2} & 0 \\ 
0 &  \frac{-1+i\sqrt{3}}{2}\\
\end{pmatrix} \begin{pmatrix}
a & b \\ 
\frac{-1-a-a^2}{b} & -1-a \\
\end{pmatrix} = \begin{pmatrix}
\frac{-1+i\sqrt{3}}{2} & \frac{-b-b i\sqrt{3}}{2} \\ 
0 & \frac{-1-i\sqrt{3}}{2} \\
\end{pmatrix} \in \Gamma_+$$
But it doesn't satisfy these conditions.
Similarly matrix with non-zero left down entry cannot exist, so all elements of $\Gamma_+$ are diagonal, and $\Gamma_+ \cong C_3$
\clm If $\Gamma$ has an element of finite order greater than 2, then it is in $\Gamma^{+}$.\\
Consider a general element of $\Gamma \setminus \Gamma^{+}$. Let its eigenvalues be $\lambda_1,\lambda_2$. Then we have  $\lambda_1+\lambda_2=k\in \Z$ and $\lambda_1\lambda_2=-1$. These imply that $\lambda_1=\frac{1}{2}\left( k-\sqrt{k^2+4}  \right) $. Thus $\sqrt{k^2+4} $ has to be an integer. This is only possible if $k=0$ in which case  $\lambda_1=\pm 1$ and $\lambda_2=\mp 1$. But this means that the element has order 2 because its square has two eigenvalues equal to 1, hence is the identity matrix. This excludes the possibilities of $\Gamma$ being $C_4,C_6,\Z\times C_4$

\\
 
Summary of nontrivial possibilities for group $\Gamma$:
\begin{enumerate}
	\item $C_2$: Given by $W(x,y) = x^2 + y^2+ x y$ 
	\item $C_2 \times C_2$: Given by $W(x,y) = x+ y +\frac{1}{x}+\frac{1}{y}$
	\item $C_3$: We haven't found a corresponding $W(x,y)$ yet
	\item $C_4$: We haven't found a corresponding $W(x,y)$ yet
	\item $S_3$: Given by $W(x,y) = x + y +\frac{1}{xy}$
	\item $C_6$: We haven't found a corresponding $W(x,y)$ yet
	\item $\mathbb{Z}$: We haven't found a corresponding $W(x,y)$ yet
	\item $\mathbb{Z} \times C_2$: We haven't found a corresponding $W(x,y)$ yet
	\item $\mathbb{Z} \times C_4$: We haven't found a corresponding $W(x,y)$ yet
	\item $\mathbb{Z} \times C_2 \times C_2$: We haven't found a corresponding $W(x,y)$ yet
	\item $D_\infty$: $W(x,y) = x^n$(not sure about this one)
	\item $D_\infty \times C_2$: We haven't found a corresponding $W(x,y)$ yet
	\item Possibly something else with index 2 subgroup isomorphic to $\mathbb{Z} \times C_2$(We're not sure if other groups exist)
\end{enumerate}

Another idea for the case of Claim 2. is to express $W(x,y) = P(x^a y^b)$ for some Laurent polynomial $P$ to get that critical points have form $x^a y^b = c$ where $P'(c) = 0$. I haven't written it up, but this should give some nice restrictions on group $\Gamma$ and exclude some of the possibilities from this list.
\end{document}
