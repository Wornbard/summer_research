\documentclass[a4paper]{article}

% Some basic packages
\pdfminorversion=7

\usepackage[utf8]{inputenc}
\usepackage[T1]{fontenc}
\usepackage{textcomp}
\usepackage[english]{babel}
\usepackage{url}
\usepackage{graphicx}
\usepackage{float}
\usepackage{booktabs}
\usepackage{enumitem}

% Don't indent paragraphs, leave some space between them
\usepackage{parskip}

% Hide page number when page is empty
\usepackage{emptypage}
\usepackage{subcaption}
\usepackage{multicol}
\usepackage{xcolor}

% Other font I sometimes use.
% \usepackage{cmbright}

% Math stuff
\usepackage{amsmath, amsfonts, mathtools, amsthm, amssymb}
% Fancy script capitals
\usepackage{mathrsfs}
\usepackage{cancel}
% Bold math
\usepackage{bm}
% Some shortcuts
\newcommand\N{\ensuremath{\mathbb{N}}}
\newcommand\R{\ensuremath{\mathbb{R}}}
\newcommand\Z{\ensuremath{\mathbb{Z}}}
\renewcommand\O{\ensuremath{\emptyset}}
\newcommand\Q{\ensuremath{\mathbb{Q}}}
\newcommand\C{\ensuremath{\mathbb{C}}}
\newcommand\mathP{\ensuremath{\mathbb{P}}}
\newcommand{\tens}[1]{%
	\mathbin{\mathop{\otimes}\limits_{#1}}%
}
\DeclareMathOperator{\Aut}{Aut}
\DeclareMathOperator{\Inn}{Inn}
\DeclareMathOperator{\Hom}{Hom}
\DeclareMathOperator{\Tr}{Tr}
% Easily typeset systems of equations (French package)
\usepackage{systeme}

% Put x \to \infty below \lim
\let\svlim\lim\def\lim{\svlim\limits}

%Make implies and impliedby shorter
\let\implies\Rightarrow
\let\impliedby\Leftarrow
\let\iff\Leftrightarrow
\let\epsilon\varepsilon

% Add \contra symbol to denote contradiction
\usepackage{stmaryrd} % for \lightning
\newcommand\contra{\scalebox{1.5}{$\lightning$}}

% \let\phi\varphi

% Command for short corrections
% Usage: 1+1=\correct{3}{2}

\definecolor{correct}{HTML}{009900}
\newcommand\correct[2]{\ensuremath{\:}{\color{red}{#1}}\ensuremath{\to }{\color{correct}{#2}}\ensuremath{\:}}
\newcommand\green[1]{{\color{correct}{#1}}}

% horizontal rule
\newcommand\hr{
    \noindent\rule[0.5ex]{\linewidth}{0.5pt}
}

% hide parts
\newcommand\hide[1]{}

% si unitx
\usepackage{siunitx}
\sisetup{locale = FR}

% Environments
\makeatother
% For box around Definition, Theorem, \ldots
\usepackage{mdframed}
\mdfsetup{skipabove=1em,skipbelow=0em}
\theoremstyle{definition}
\newmdtheoremenv[nobreak=true]{definitie}{Definitie}
\newmdtheoremenv[nobreak=true]{eigenschap}{Eigenschap}
\newmdtheoremenv[nobreak=true]{gevolg}{Gevolg}
\newmdtheoremenv[nobreak=true]{lemma}{Lemma}
\newmdtheoremenv[nobreak=true]{propositie}{Propositie}
\newmdtheoremenv[nobreak=true]{definition}{Definition}
\newtheorem*{eg}{Example}
\newtheorem*{notation}{Notation}
\newtheorem*{previouslyseen}{As previously seen}
\newtheorem*{remark}{Remark}
\newtheorem*{note}{Note}
\newtheorem*{problem}{Problem}
\newtheorem*{observe}{Observe}
\newtheorem*{property}{Property}
\newtheorem*{intuition}{Intuition}
\newmdtheoremenv[nobreak=true]{prop}{Proposition}
\newmdtheoremenv[nobreak=true]{theorem}{Theorem}
\newmdtheoremenv[nobreak=true]{corollary}{Corollary}

% End example and intermezzo environments with a small diamond (just like proof
% environments end with a small square)
\usepackage{etoolbox}
\AtEndEnvironment{vb}{\null\hfill$\diamond$}%
\AtEndEnvironment{intermezzo}{\null\hfill$\diamond$}%
% \AtEndEnvironment{opmerking}{\null\hfill$\diamond$}%

% Fix some spacing
% http://tex.stackexchange.com/questions/22119/how-can-i-change-the-spacing-before-theorems-with-amsthm
\makeatletter
\def\thm@space@setup{%
  \thm@preskip=\parskip \thm@postskip=0pt
}


% \lecture starts a new lecture (les in dutch)
%
% Usage:
% \lecture{1}{di 12 feb 2019 16:00}{Inleiding}
%
% This adds a section heading with the number / title of the lecture and a
% margin paragraph with the date.

% I use \dateparts here to hide the year (2019). This way, I can easily parse
% the date of each lecture unambiguously while still having a human-friendly
% short format printed to the pdf.

\usepackage{xifthen}
\def\testdateparts#1{\dateparts#1\relax}
\def\dateparts#1 #2 #3 #4 #5\relax{
    \marginpar{\small\textsf{\mbox{#1 #2 #3 #5}}}
}

\def\@lecture{}%
\newcommand{\lecture}[3]{
    \ifthenelse{\isempty{#3}}{%
        \def\@lecture{Lecture #1}%
    }{%
        \def\@lecture{Lecture #1: #3}%
    }%
    \subsection*{\@lecture}
    \marginpar{\small\textsf{\mbox{#2}}}
}

\DeclareMathOperator{\Ima}{Im}

% These are the fancy headers
\usepackage{fancyhdr}
\pagestyle{fancy}

% LE: left even
% RO: right odd
% CE, CO: center even, center odd
% My name for when I print my lecture notes to use for an open book exam.
% \fancyhead[LE,RO]{Gilles Castel}

\fancyhead[RO,LE]{\@lecture} % Right odd,  Left even
\fancyhead[RE,LO]{}          % Right even, Left odd

\fancyfoot[RO,LE]{\thepage}  % Right odd,  Left even
\fancyfoot[RE,LO]{}          % Right even, Left odd
\fancyfoot[C]{\leftmark}     % Center

\makeatother




% Todonotes and inline notes in fancy boxes
\usepackage{todonotes}
\usepackage{tcolorbox}

% Make boxes breakable
\tcbuselibrary{breakable}

% Figure support as explained in my blog post.
\usepackage{import}
\usepackage{xifthen}
\usepackage{pdfpages}
\usepackage{transparent}
\newcommand{\incfig}[1]{%
    \def\svgwidth{\columnwidth}
    \import{./figures/}{#1.pdf_tex}
}

% Fix some stuff
% %http://tex.stackexchange.com/questions/76273/multiple-pdfs-with-page-group-included-in-a-single-page-warning
\pdfsuppresswarningpagegroup=1


% My name
\author{Jakub Wornbard}


\usepackage[utf8]{inputenc}
\usepackage[T1]{fontenc}
\usepackage{textcomp}
\usepackage[english]{babel}
\usepackage{amsmath, amssymb}
\newtheorem{thm}{Theorem}
\newtheorem{lem}[thm]{Lemma}
\newtheorem{exmp}[thm]{Example}
\newtheorem{defn}[thm]{Definition}
\newtheorem{que}[thm]{Question}
\newtheorem{clm}[thm]{Claim}
\pdfsuppresswarningpagegroup=1

\begin{document}
	Let $M \subset Z^2$ be the set of exponents of monomials in $W(x,y)$.
	
\clm If $M$ contains two linearly independent vectors, then every element in $\Gamma$ has finite order.

 Suppose that $M$ contains two linearly independent vectors $x$ and $y$. Then for any $A \in \Gamma$ we have $A^k x= x$ for some $k$ (because $M$ is finite). Similarly $A^m y = y$ for some $y$. Hence $A^{k m}$ fixes both $x$ and $y$, so $A^{k m} = I$, so $A$ has finite order.
\clm If $M$ contains no two independent vectors, then $\Gamma_+ \cong \mathbb{Z}$ or $\Gamma_+ \cong \mathbb{Z} \times C_{2}$ or $\Gamma_+ \cong C_{2}$.

Let $(a,b) \in M$ be the vector with minimal length.
Choose a basis for $Z^2$ such that $(a,b)$ is a multiple of $(1,0)$ in a new basis. Then $(1,0)$ is an eigenvector of $A$ for any $A \in \Gamma_+$. Moreover, the eigenvalue is either $1$ or $-1$, because otherwise we could obtain an infinite sequence of distinct monomials. For any $A \in \Gamma_+$ we also need the determinant to be 1, so it either has the form:
$\begin{pmatrix}
	1 & a \\
	0 & 1 \\
\end{pmatrix}$ or the form 
$\begin{pmatrix}
-1 & b \\
0 & -1 \\
\end{pmatrix}$
for some $a, b$ in $\mathbb{Z}$.

$\begin{pmatrix}
1 & 1 \\ 
0 & 1 \\
\end{pmatrix}$ or the form 
$\begin{pmatrix}
-1 & 1 \\
0 & -1 \\
\end{pmatrix}$ generate a group isomorphic to $\mathbb{Z} \times C_2$, and $\Gamma_+$ is its subgroup so $\Gamma_+ \cong \mathbb{Z}$ or $\Gamma_+ \cong \mathbb{Z} \times C_2$ or $\Gamma_+ \cong C_2$
\clm If $M$ contains two linearly independent vectors, then $\Gamma_+ \cong C_2$ or $\Gamma_+ \cong C_3$
Suppose that $\Gamma_+$ contains an element of order 2. It has determinant $1$, so it either has eigenvalues $1, 1$ or $-1, -1$. Direct check of conjugacy classes shows that it has to be conjugate to $\begin{pmatrix}
-1 & 0 \\ 
0 & -1 \\
\end{pmatrix}$, hence equal to it. We know that then $\Gamma_+$ has no other elements of finite order, so no other elements at all, hence $\Gamma_+ \equiv C_2$.
The only other possibility is that $\Gamma_+$ has exponent 3. Then by considering 3rd roots of unity we see that any non-identity element $A$ in $\Gamma_+$ has eigenvalues $\frac{-1-i\sqrt{3}}{2}$ and $\frac{-1+i\sqrt{3}}{2}$. Hence, we can pick an appropriate basis and assume that $\begin{pmatrix}
\frac{-1-i\sqrt{3}}{2} & 0 \\ 
0 &  \frac{-1+i\sqrt{3}}{2}\\
\end{pmatrix} \in \Gamma_+$.In particular, any non-identity element of $A$ has trace $-1$ and determinant $1$. Suppose $\Gamma_+$ contains a matrix with non-zero top right entry. It then has the form:
$\begin{pmatrix}
a & b \\ 
\frac{-1-a-a^2}{b} & -1-a \\
\end{pmatrix}$ 
We need 
$$\begin{pmatrix}
\frac{-1-i\sqrt{3}}{2} & 0 \\ 
0 &  \frac{-1+i\sqrt{3}}{2}\\
\end{pmatrix} \begin{pmatrix}
	a & b \\ 
	\frac{-1-a-a^2}{b} & -1-a \\
\end{pmatrix}$$ to have trace $-1$. Hence $a = \frac{-1-i\sqrt{3}}{2}$, and so left down entry is $0$. 
But then 
$$\begin{pmatrix}
\frac{-1-i\sqrt{3}}{2} & 0 \\ 
0 &  \frac{-1+i\sqrt{3}}{2}\\
\end{pmatrix} \begin{pmatrix}
a & b \\ 
\frac{-1-a-a^2}{b} & -1-a \\
\end{pmatrix} = \begin{pmatrix}
\frac{-1+i\sqrt{3}}{2} & \frac{-b-b i\sqrt{3}}{2} \\ 
0 & \frac{-1-i\sqrt{3}}{2} \\
\end{pmatrix} \in \Gamma_+$$
But it doesn't satisfy these conditions.
Similarly matrix with non-zero left down entry cannot exist, so all elements of $\Gamma_+$ are diagonal, and $\Gamma_+ \cong C_3$

 
Summary of possibilities for group $\Gamma$:
\begin{enumerate}
	\item $C_2$: Given by $W(x,y) = x^2 + y^2+ x y$ 
	\item $C_2 \times C_2$: Given by $W(x,y) = x+ y +\frac{1}{x}+\frac{1}{y}$
	\item $C_3$: We haven't found a corresponding $W(x,y)$ yet
	\item $C_4$: We haven't found a corresponding $W(x,y)$ yet
	\item $S_3$: Given by $W(x,y) = x + y +\frac{1}{xy}$
	\item $C_6$: We haven't found a corresponding $W(x,y)$ yet
	\item $\mathbb{Z}$: We haven't found a corresponding $W(x,y)$ yet
	\item $\mathbb{Z} \times C_2$: $W(x,y) = x^n$
	\item $\mathbb{Z} \times C_4$: We haven't found a corresponding $W(x,y)$ yet
	\item $\mathbb{Z} \times C_2 \times C_2$: We haven't found a corresponding $W(x,y)$ yet
	\item $D_\infty$: We haven't found a corresponding $W(x,y)$ yet
	\item Possibly something else with index 2 subgroup isomorphic to $\mathbb{Z}$ or $\mathbb{Z} \times C_2$(We're not sure if other groups exist)
\end{enumerate}
\end{document}
