\documentclass[a4paper]{article}

\input{preamble.tex}
\usepackage[utf8]{inputenc}
\usepackage[T1]{fontenc}
\usepackage{textcomp}
\usepackage[english]{babel}
\usepackage{amsmath, amssymb}
\newtheorem{thm}{Theorem}
\newtheorem{lem}[thm]{Lemma}
\newtheorem{exmp}[thm]{Example}
\newtheorem{defn}[thm]{Definition}
\newtheorem{que}[thm]{Question}
\pdfsuppresswarningpagegroup=1

\begin{document}

After doing all the computations from the meeting we end up with a quadratic form given by the matrix $Q=\begin{pmatrix} -1 & -\frac{1}{2}\\-\frac{1}{2} & -1 \end{pmatrix} $ and we are looking for an inner automorphism of the corresponding Clifford algebra that sends $e_1$ to $e_2$ and $e_2$ to $-e_1-e_2$ where $e_1, e_2$ are basis vectors. We would like to work in a nicer representation of this algebra. 
Taking the matrix $A=\begin{pmatrix} \frac{i\sqrt{3} }{2} & \frac{i\sqrt{3} }{2}\\-\frac{i}{2}&\frac{i}{2} \end{pmatrix} $ we note that $A^{T}A=Q$ so the identity quadratic form evaluated on $Av$ equals $Q$ evaluated on $v$. Hence in basis $f_1 = \frac{i\sqrt{3}}{2}e_1-\frac{i}{2}e_2$, $f_2 = \frac{i\sqrt{3} }{2}e_1+\frac{i}{2}e_2$ the quadratic form has the standard form $Q(x) = x_1^2 +x_2^2$. We have:
$$e_1 = -\frac{i\sqrt{3}}{3}(f_1+f_2)$$
$$e_2 = i(f_1 - f_2)$$
So $e_1e_2 = \frac{\sqrt{3}}{3}(f_1^2-f_2^2-f_1f_2+f_2f_1)= -2\frac{\sqrt{3}}{3}f_1f_2$
Hence an arbitrary element $a + b e_1 + c e_2 + d e_1e_2$ can be written as
$$a+(i c - \frac{i b}{\sqrt{3}}) f_1 +(-i c - \frac{i b}{\sqrt{3}}) f_2 - \frac{2d\sqrt{3}}{3}f_1f_2$$
Over a basis $f_1, f_2$ our algebra is naturally isomorphic to biquaternions, hence elements are invertible precisely when sum of squares of the coefficients is non-zero. Hence $a + b e_1 + c e_2 + d e_1e_2$ is invertible iff 
$$a^2+(i c - \frac{i b}{\sqrt{3}})^2 +(-i c - \frac{i b}{\sqrt{3}})^2 - (\frac{2d\sqrt{3}}{3})^2 = a^2-c^2-\frac{2}{3}(b^2-2d^2)$$ is nonzero.
Also under this change of basis scalars are bijectively mapped to scalars and centre of biquaternions are precisely scalars, so an element belongs to the centre iff it is a scalar.
Hence two elements give the same inner automorphism iff they are scalar multiples of each other.

Suppose that $x = a + b e_1 + c e_2 + d e_1 e_2$ gives the automorphism we want. Then $x e_2 = e_1 x$ and $x (-e_1 - e_2) = e_2 x$.
Expanding the first one we get:
$$(a + b e_1 + c e_2 + d e_1 e_2) e_2 = e_1 (a + b e_1 + c e_2 + d e_1 e_2)$$
$$a e_2 + b e_1 e_2 - c - d e_1 = e_1 a - b +c e_1 e_2 - d e_2$$
So we get $b=c$ and $a = -d$.
Expanding the second condition we get:	
$$(a + b e_1 + b e_2  -a e_1 e_2)(-e_1- e_2) = e_2 (a + b e_1 + b e_2 - a e_1 e_2)$$
$$-a e_1- a e_2 +3 b - a e_1 +a e_1(e_2 e_1) = a e_2 +b e_2 e_1 - b - a e_2 e_1 e_2$$
$$-a e_1- a e_2 +3 b - a e_1 +a e_1(-1-e_1 e_2) = a e_2 +b (-1-e_1 e_2) - b - a (-1-e_1e_2)e_2$$
$$-2 a e_1- a e_2 +3 b - a e_1 +a e_2 = a e_2  -b e_1 e_2 - 2 b +a e_2 - a e_1 $$
So $a=b=0$, so there should be no such element.
\end{document}
