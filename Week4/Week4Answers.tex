\documentclass[a4paper]{article}

\input{preamble.tex}
\usepackage[utf8]{inputenc}
\usepackage[T1]{fontenc}
\usepackage{textcomp}
\usepackage[english]{babel}
\usepackage{amsmath, amssymb}
\newtheorem{thm}{Theorem}
\newtheorem{lem}[thm]{Lemma}
\newtheorem{exmp}[thm]{Example}
\newtheorem{defn}[thm]{Definition}
\newtheorem{que}[thm]{Question}
\pdfsuppresswarningpagegroup=1

\begin{document}

After doing all the computations from the meeting we end up with a quadratic form given by the matrix $Q=\begin{pmatrix} -1 & -\frac{1}{2}\\-\frac{1}{2} & -1 \end{pmatrix} $ and we are looking for an inner automorphism of the corresponding Clifford algebra that sends $e_1$ to $e_2$ and $e_2$ to $-e_1-e_2$ where $e_1, e_2$ are basis vectors. We would like to work in a nicer representation of this algebra. 
Taking the matrix $A=\begin{pmatrix} \frac{i\sqrt{3} }{2} & \frac{i\sqrt{3} }{2}\\-\frac{i}{2}&\frac{i}{2} \end{pmatrix} $ we note that $A^{T}A=Q$ so the identity quadratic form evaluated on $Av$ equals $Q$ evaluated on $v$. Hence in basis $f_1 = \frac{i\sqrt{3}}{2}e_1-\frac{i}{2}e_2$, $f_2 = \frac{i\sqrt{3} }{2}e_1+\frac{i}{2}e_2$ the quadratic form has the standard form $Q(x) = x_1^2 +x_2^2$. We have:
$$e_1 = -\frac{i\sqrt{3}}{3}(f_1+f_2)$$
$$e_2 = i(f_1 - f_2)$$
So $e_1e_2 = \frac{\sqrt{3}}{3}(f_1^2-f_2^2-f_1f_2+f_2f_1)= -2\frac{\sqrt{3}}{3}f_1f_2$
Hence an arbitrary element $a + b e_1 + c e_2 + d e_1e_2$ can be written as
$$a+(i c - \frac{i b}{\sqrt{3}}) f_1 +(-i c - \frac{i b}{\sqrt{3}}) f_2 - \frac{2d\sqrt{3}}{3}f_1f_2$$
Over a basis $f_1, f_2$ our algebra is naturally isomorphic to biquaternions, hence elements are invertible precisely when sum of squares of the coefficients is non-zero. Hence $a + b e_1 + c e_2 + d e_1e_2$ is invertible iff 
$$a^2+(i c - \frac{i b}{\sqrt{3}})^2 +(-i c - \frac{i b}{\sqrt{3}})^2 - (\frac{2d\sqrt{3}}{3})^2 = a^2-c^2-\frac{2}{3}(b^2-2d^2)$$ is nonzero.
Hence in our "standard" Clifford algebra instead of our earlier mappings we now want to map the images under $A$ that is in terms of its basis vectors we want $\frac{i\sqrt{3}}{2}e_1-\frac{i}{2}e_2\mapsto \frac{i\sqrt{3} }{2}e_1+\frac{i}{2}e_2$ and $\frac{i\sqrt{3} }{2}e_1+\frac{i}{2}e_2\mapsto -i\sqrt{3}e_1 $
This is equivalent to $e_1\mapsto -\frac{1}{2}e_1+\frac{1}{2\sqrt{3} }e_{2}$ and $e_2\to -\frac{3\sqrt{3} }{2}e_1-\frac{1}{2}e_2$

We were told that all the automorphisms of our algebra are inner so there is not much to consider. The algebra is generated by $e_1,e_2$ subject to $e_1^2=e_2^2=1$ so a direct check shows that $e_1,e_2, e_1e_2$ are not in the centre of the algebra. Hence the centre only consists of scalars.  Now let's suppose that for elements $g,h$, we have for all  $v$ :  $g^{-1}vg=h^{-1}vh$. Then $hg^{-1}$ is in the centre, so it's a scalar. Let it equal $\lambda$. Note it can't be 0 because then  $h$ would be a zero-divisor but it's a unit so this can't be the case. Hence $\lambda$ is non-zero so invertible. Thus $h\left( \lambda g \right) ^{-1}=1$ so $h$ and $g$ only differ by a scalar. Hence our automorphism is unique (up to congujating by a scalar but this does nothing).

\end{document}
