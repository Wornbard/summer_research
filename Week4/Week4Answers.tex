\documentclass[a4paper]{article}

% Some basic packages
\pdfminorversion=7

\usepackage[utf8]{inputenc}
\usepackage[T1]{fontenc}
\usepackage{textcomp}
\usepackage[english]{babel}
\usepackage{url}
\usepackage{graphicx}
\usepackage{float}
\usepackage{booktabs}
\usepackage{enumitem}

% Don't indent paragraphs, leave some space between them
\usepackage{parskip}

% Hide page number when page is empty
\usepackage{emptypage}
\usepackage{subcaption}
\usepackage{multicol}
\usepackage{xcolor}

% Other font I sometimes use.
% \usepackage{cmbright}

% Math stuff
\usepackage{amsmath, amsfonts, mathtools, amsthm, amssymb}
% Fancy script capitals
\usepackage{mathrsfs}
\usepackage{cancel}
% Bold math
\usepackage{bm}
% Some shortcuts
\newcommand\N{\ensuremath{\mathbb{N}}}
\newcommand\R{\ensuremath{\mathbb{R}}}
\newcommand\Z{\ensuremath{\mathbb{Z}}}
\renewcommand\O{\ensuremath{\emptyset}}
\newcommand\Q{\ensuremath{\mathbb{Q}}}
\newcommand\C{\ensuremath{\mathbb{C}}}
\newcommand\mathP{\ensuremath{\mathbb{P}}}
\newcommand{\tens}[1]{%
	\mathbin{\mathop{\otimes}\limits_{#1}}%
}
\DeclareMathOperator{\Aut}{Aut}
\DeclareMathOperator{\Inn}{Inn}
\DeclareMathOperator{\Hom}{Hom}
\DeclareMathOperator{\Tr}{Tr}
% Easily typeset systems of equations (French package)
\usepackage{systeme}

% Put x \to \infty below \lim
\let\svlim\lim\def\lim{\svlim\limits}

%Make implies and impliedby shorter
\let\implies\Rightarrow
\let\impliedby\Leftarrow
\let\iff\Leftrightarrow
\let\epsilon\varepsilon

% Add \contra symbol to denote contradiction
\usepackage{stmaryrd} % for \lightning
\newcommand\contra{\scalebox{1.5}{$\lightning$}}

% \let\phi\varphi

% Command for short corrections
% Usage: 1+1=\correct{3}{2}

\definecolor{correct}{HTML}{009900}
\newcommand\correct[2]{\ensuremath{\:}{\color{red}{#1}}\ensuremath{\to }{\color{correct}{#2}}\ensuremath{\:}}
\newcommand\green[1]{{\color{correct}{#1}}}

% horizontal rule
\newcommand\hr{
    \noindent\rule[0.5ex]{\linewidth}{0.5pt}
}

% hide parts
\newcommand\hide[1]{}

% si unitx
\usepackage{siunitx}
\sisetup{locale = FR}

% Environments
\makeatother
% For box around Definition, Theorem, \ldots
\usepackage{mdframed}
\mdfsetup{skipabove=1em,skipbelow=0em}
\theoremstyle{definition}
\newmdtheoremenv[nobreak=true]{definitie}{Definitie}
\newmdtheoremenv[nobreak=true]{eigenschap}{Eigenschap}
\newmdtheoremenv[nobreak=true]{gevolg}{Gevolg}
\newmdtheoremenv[nobreak=true]{lemma}{Lemma}
\newmdtheoremenv[nobreak=true]{propositie}{Propositie}
\newmdtheoremenv[nobreak=true]{definition}{Definition}
\newtheorem*{eg}{Example}
\newtheorem*{notation}{Notation}
\newtheorem*{previouslyseen}{As previously seen}
\newtheorem*{remark}{Remark}
\newtheorem*{note}{Note}
\newtheorem*{problem}{Problem}
\newtheorem*{observe}{Observe}
\newtheorem*{property}{Property}
\newtheorem*{intuition}{Intuition}
\newmdtheoremenv[nobreak=true]{prop}{Proposition}
\newmdtheoremenv[nobreak=true]{theorem}{Theorem}
\newmdtheoremenv[nobreak=true]{corollary}{Corollary}

% End example and intermezzo environments with a small diamond (just like proof
% environments end with a small square)
\usepackage{etoolbox}
\AtEndEnvironment{vb}{\null\hfill$\diamond$}%
\AtEndEnvironment{intermezzo}{\null\hfill$\diamond$}%
% \AtEndEnvironment{opmerking}{\null\hfill$\diamond$}%

% Fix some spacing
% http://tex.stackexchange.com/questions/22119/how-can-i-change-the-spacing-before-theorems-with-amsthm
\makeatletter
\def\thm@space@setup{%
  \thm@preskip=\parskip \thm@postskip=0pt
}


% \lecture starts a new lecture (les in dutch)
%
% Usage:
% \lecture{1}{di 12 feb 2019 16:00}{Inleiding}
%
% This adds a section heading with the number / title of the lecture and a
% margin paragraph with the date.

% I use \dateparts here to hide the year (2019). This way, I can easily parse
% the date of each lecture unambiguously while still having a human-friendly
% short format printed to the pdf.

\usepackage{xifthen}
\def\testdateparts#1{\dateparts#1\relax}
\def\dateparts#1 #2 #3 #4 #5\relax{
    \marginpar{\small\textsf{\mbox{#1 #2 #3 #5}}}
}

\def\@lecture{}%
\newcommand{\lecture}[3]{
    \ifthenelse{\isempty{#3}}{%
        \def\@lecture{Lecture #1}%
    }{%
        \def\@lecture{Lecture #1: #3}%
    }%
    \subsection*{\@lecture}
    \marginpar{\small\textsf{\mbox{#2}}}
}

\DeclareMathOperator{\Ima}{Im}

% These are the fancy headers
\usepackage{fancyhdr}
\pagestyle{fancy}

% LE: left even
% RO: right odd
% CE, CO: center even, center odd
% My name for when I print my lecture notes to use for an open book exam.
% \fancyhead[LE,RO]{Gilles Castel}

\fancyhead[RO,LE]{\@lecture} % Right odd,  Left even
\fancyhead[RE,LO]{}          % Right even, Left odd

\fancyfoot[RO,LE]{\thepage}  % Right odd,  Left even
\fancyfoot[RE,LO]{}          % Right even, Left odd
\fancyfoot[C]{\leftmark}     % Center

\makeatother




% Todonotes and inline notes in fancy boxes
\usepackage{todonotes}
\usepackage{tcolorbox}

% Make boxes breakable
\tcbuselibrary{breakable}

% Figure support as explained in my blog post.
\usepackage{import}
\usepackage{xifthen}
\usepackage{pdfpages}
\usepackage{transparent}
\newcommand{\incfig}[1]{%
    \def\svgwidth{\columnwidth}
    \import{./figures/}{#1.pdf_tex}
}

% Fix some stuff
% %http://tex.stackexchange.com/questions/76273/multiple-pdfs-with-page-group-included-in-a-single-page-warning
\pdfsuppresswarningpagegroup=1


% My name
\author{Jakub Wornbard}


\usepackage[utf8]{inputenc}
\usepackage[T1]{fontenc}
\usepackage{textcomp}
\usepackage[english]{babel}
\usepackage{amsmath, amssymb}
\newtheorem{thm}{Theorem}
\newtheorem{lem}[thm]{Lemma}
\newtheorem{exmp}[thm]{Example}
\newtheorem{defn}[thm]{Definition}
\newtheorem{que}[thm]{Question}
\pdfsuppresswarningpagegroup=1

\begin{document}

After doing all the computations from the meeting we end up with a quadratic form given by the matrix $Q=\begin{pmatrix} -1 & -\frac{1}{2}\\-\frac{1}{2} & -1 \end{pmatrix} $ and we are looking for an inner automorphism of the corresponding Clifford algebra that sends $e_1$ to $e_2$ and $e_2$ to $-e_1-e_2$ where $e_1, e_2$ are basis vectors. We would like to work in a nicer representation of this algebra. 
Taking the matrix $A=\begin{pmatrix} \frac{i\sqrt{3} }{2} & \frac{i\sqrt{3} }{2}\\-\frac{i}{2}&\frac{i}{2} \end{pmatrix} $ we note that $A^{T}A=Q$ so the identity quadratic form evaluated on $Av$ equals $Q$ evaluated on $v$. Hence in basis $f_1 = \frac{i\sqrt{3}}{2}e_1-\frac{i}{2}e_2$, $f_2 = \frac{i\sqrt{3} }{2}e_1+\frac{i}{2}e_2$ the quadratic form has the standard form $Q(x) = x_1^2 +x_2^2$. We have:
$$e_1 = -\frac{i\sqrt{3}}{3}(f_1+f_2)$$
$$e_2 = i(f_1 - f_2)$$
So $e_1e_2 = \frac{\sqrt{3}}{3}(f_1^2-f_2^2-f_1f_2+f_2f_1)= -2\frac{\sqrt{3}}{3}f_1f_2$
Hence an arbitrary element $a + b e_1 + c e_2 + d e_1e_2$ can be written as
$$a+(i c - \frac{i b}{\sqrt{3}}) f_1 +(-i c - \frac{i b}{\sqrt{3}}) f_2 - \frac{2d\sqrt{3}}{3}f_1f_2$$
Over a basis $f_1, f_2$ our algebra is naturally isomorphic to biquaternions, hence elements are invertible precisely when sum of squares of the coefficients is non-zero. Hence $a + b e_1 + c e_2 + d e_1e_2$ is invertible iff 
$$a^2+(i c - \frac{i b}{\sqrt{3}})^2 +(-i c - \frac{i b}{\sqrt{3}})^2 - (\frac{2d\sqrt{3}}{3})^2 = a^2-c^2-\frac{2}{3}(b^2-2d^2)$$ is nonzero.
Also under this change of basis scalars are bijectively mapped to scalars and centre of biquaternions are precisely scalars, so an element belongs to the centre iff it is a scalar.
Hence two elements give the same inner automorphism iff they are scalar multiples of each other.

Suppose that $x = a + b e_1 + c e_2 + d e_1 e_2$ gives the automorphism we want. Then $x e_2 = e_1 x$ and $x (-e_1 - e_2) = e_2 x$.
Expanding the first one we get:
$$(a + b e_1 + c e_2 + d e_1 e_2) e_2 = e_1 (a + b e_1 + c e_2 + d e_1 e_2)$$
$$a e_2 + b e_1 e_2 - c - d e_1 = e_1 a - b +c e_1 e_2 - d e_2$$
So we get $b=c$ and $a = -d$.
Expanding the second condition we get:	
$$(a + b e_1 + b e_2  -a e_1 e_2)(-e_1- e_2) = e_2 (a + b e_1 + b e_2 - a e_1 e_2)$$
$$-a e_1- a e_2 +3 b - a e_1 +a e_1(e_2 e_1) = a e_2 +b e_2 e_1 - b - a e_2 e_1 e_2$$
$$-a e_1- a e_2 +3 b - a e_1 +a e_1(-1-e_1 e_2) = a e_2 +b (-1-e_1 e_2) - b - a (-1-e_1e_2)e_2$$
$$-2 a e_1- a e_2 +3 b - a e_1 +a e_2 = a e_2  -b e_1 e_2 - 2 b +a e_2 - a e_1 $$
So $a=b=0$, so there should be no such element.
\end{document}
