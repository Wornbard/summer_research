\documentclass[a4paper]{article}

\input{preamble.tex}
\usepackage{tikz-cd}
\usepackage[utf8]{inputenc}
\usepackage[T1]{fontenc}
\usepackage{textcomp}
\usepackage[english]{babel}
\usepackage{amsmath, amssymb}
\newtheorem{thm}{Theorem}
\newtheorem{lem}[thm]{Lemma}
\newtheorem{exmp}[thm]{Example}                                                                 
\newtheorem{defn}[thm]{Definition}
\newtheorem{que}[thm]{Question}   

\pdfsuppresswarningpagegroup=1

\begin{document}
\section*{ 2020-06-18}

\begin{que} Can you embed the following groups into $F_2$?
	\begin{itemize}
		\item $F_3$ 
		
		Same as $F_\infty$, but restricted to first three generators.
		\item $F_\infty$ 
		
		Let $a, b$ be generators of $F_2$. Consider $x_i = b^{-i} a b^{-i}$ for $i = 1, 2, ...$.
			Let's prove that $\left< x_1, x_2, ... \right>$ is free:
			Suppose that some reduced word $w = x_{y_1}^{z_1} x_{y_2}^{z_2} ...$ is equal to identity.
			It is reduced, so $y_i \neq y_i+1$ for any $i$	
			$$b^{-y_1} a^{z_1} b^{y_1- y_2} a^{z_2} b^{y_2 - y_3} ... = e$$
			All exponents in the above word are non-zero, so the above word is reduced and non-empty, so it is not equal to identity, contradiction.
			Hence $\left< x_1, x_2, ... \right> \cong F_{\infty}$, so $F_2$ has a subgroup isomorphic to $F_{\infty}$.
			
	\end{itemize}
\end{que} 
\begin{que} How to recover $d_i$'s in classification of finitely generated abelian groups?
	
\end{que}
\begin{que} Show that free R-modules are projective.
	
	Let $p_1, ... p_k$ be a basis for a free R-module $P$. Suppose $f : N \to M$ is a surjective homomorphism and $g: N \to M$ is a homomorphism. $f$ is surjective, so let $n_i$ be s.t. $f(n_i) = g(p_i)$. 
	Now let $h : P \to M$ be defined by $h(r_i p_i)=r_i n_i$. This is well defined, because $P$ is free, and it's also a homeomorphism. Moreover $f \circ h(r_i p_i) = f(r_i n_i) = r_i g(p_i) = g(r_i p_i)$, so $g = f \circ h$, so $P$ is projective.
\end{que}
\begin{que} Show that if P is projective then $\otimes P$ preserves injectivity of maps.
\end{que}
\begin{que} Show that $\Hom_{S-mod}(S \tens{R} M, N) = \Hom_{R-mod}(M, N_R)$.
\end{que}
\begin{que} Check why dividing by $\{m \otimes m : m \in M\}$ implies that $m_1 \otimes m_2 + m_2 \otimes m_1 = 0$ but the opposite doesn't hold in characteristic $2$
\end{que}
\begin{que} Assume $M$ is free of rank $n$. What is $\Lambda^i M$?

\end{que}
\begin{que} Given $f : M \to M$ a homomorphism of R-modules, write down two "interesting" maps
$$\Lambda^i M \to \Lambda^i M$$ induced by $f$. Interpret these maps when $i=n$.

Let $g_1 : \Lambda^i M \to \Lambda^i M$ be defined by $g_1(m_1 \otimes ... \otimes m_i) = f(m_1) \otimes ... f(m_i)$. 
Let $g_2 : \Lambda^i M \to \Lambda^i M$ be defined by $g_2(m_1 \otimes ... \otimes m_i) = f(m_1) \otimes ... \otimes m_i + m_1 \otimes f(m_2) ... \otimes m_i + ... + m_1 \otimes m_2 ... \otimes f(m_i)$. They are well defined because they factor through the quotient defining the exterior algebra.
If $i=n$, let $e_1, e_2, ..., e_n$ be the basis of $M$, and let $f_ij$ be the j-th component of $f(e_i)$ in this basis. The only basis vector for $\Lambda^n M$ is $e_1 \otimes e_2 ... \otimes e_n$:
$$g_1(C e_1 \otimes e_2 ... \otimes e_n) = C f(e_1) \otimes (e_2) ... \otimes f(e_n) = C (f_{1j}e_j) \otimes (f_{2j}e_j) ... \otimes (f_{nj}e_j) = C \det(f_{ij})  e_1 \otimes e_2 ... \otimes e_n$$
So $g_1$ corresponds to multiplication by $\det(f_{ij})$.
Similarly:
$$g_2(C( e_1 \otimes e_2 ... \otimes e_n)) = C(f(e_1) \otimes ... \otimes e_n + e_1 \otimes f(e_2) ... \otimes e_i + ... + e_1 \otimes e_2 ... \otimes f(e_n))$$
$$ = C(e_{11} + e_{22}+...+e_{nn}) e_1 \otimes e_2 ... \otimes e_n = C \Tr(f_{ij}) e_1 \otimes e_2 ... \otimes e_n$$
So $g_2$ corresponds to multiplication by $\Tr(f_{ij})$.
\end{que}

\end{document}
