\documentclass[a4paper]{article}

\input{preamble.tex}
\usepackage{tikz-cd}
\usepackage[utf8]{inputenc}
\usepackage[T1]{fontenc}
\usepackage{textcomp}
\usepackage[english]{babel}
\usepackage{amsmath, amssymb}
\newtheorem{thm}{Theorem}
\newtheorem{lem}[thm]{Lemma}
\newtheorem{exmp}[thm]{Example}                                                                 
\newtheorem{defn}[thm]{Definition}
\newtheorem{que}[thm]{Question}   

\pdfsuppresswarningpagegroup=1

\begin{document}
\section*{ 2020-06-18}

\begin{que} Can you embed the following groups into $F_2$?
	\begin{itemize}
		\item $F_3$
		\item $F_\infty$
	\end{itemize}
\end{que} 
\begin{que} How to recover $d_i$'s in classification of finitely generated abelian groups?
	
\end{que}
\begin{que} Show that free R-modules are projective
\end{que}
\begin{que} Show that if P is projective then $\otimes P$ preserves injectivity of maps.
\end{que}
\begin{que} Show that $\Hom_{S-mod}(S \tens{R} M, N) = \Hom_{R-mod}(M, N_R)$.
\end{que}
\begin{que} Check why dividing by $\{m \otimes m : m \in M\}$ implies that $m_1 \otimes m_2 + m_2 \otimes m_1 = 0$ but the opposite doesn't hold in characteristic $2$
\end{que}
\begin{que} Assume $M$ is free of rank $n$. What is $\Lambda^i M$?

\end{que}
\begin{que} Given $f : M \to M$ a homomorphism of R-modules, write down two "interesting" maps
$$\Lambda^i M \to \Lambda^i M$$ induced by $f$. Interpret these maps when $i=n$.
\end{que}

\end{document}
