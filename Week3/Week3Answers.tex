\documentclass[a4paper]{article}

\input{preamble.tex}
\usepackage{tikz-cd}
\usepackage[utf8]{inputenc}
\usepackage[T1]{fontenc}
\usepackage{textcomp}
\usepackage[english]{babel}
\usepackage{amsmath, amssymb}
\newtheorem{thm}{Theorem}
\newtheorem{lem}[thm]{Lemma}
\newtheorem{exmp}[thm]{Example}                                                                 
\newtheorem{defn}[thm]{Definition}
\newtheorem{que}[thm]{Question}   

\pdfsuppresswarningpagegroup=1

\begin{document}
\section*{ 2020-06-25}

\begin{que} Find a space $Y$ on which $F_2/F_\infty$ acts such that the quotient is wedge of two circles.

\end{que} 
\begin{que} For the deformed product to be associative, what properties must $m_1$ have?
	
	Keeping terms up to order $t$ we have:
	$$m(m(r_1, r_2), r_3) = m_0(m_0(r_1, r_2) + t m_1(r_1,r_2), r_3) + t m_1(m(r_1, r_2), r_3)) = $$
	$$ = m_0(m_0(r_1, r_2), r_3) + t (m_0(m_1(r_1, r_2), r_3)) +  m_1(m_0(r_1, r_2), r_3))$$
	And similarly for $m(r_1, m(r_2, r_3))$. Hence we need 
	$$m_0(m_1(r_1, r_2), r_3)) +  m_1(m_0(r_1, r_2), r_3) = m_0(r_1, m_1(r_2, r_3)) +  m_1(r_1, m_0(r_2, r_3))$$.
\end{que}
\begin{que} Classify quadratic forms up to isomorphism when $K = \mathbb{C}$ or $K = \mathbb{R}$. 
	
	
	Clearly any real quadratic form $Q$ can be (uniquely) represented by a symmetric matrix $B$ such that
	$$Q(x) = x^T B x$$
	Any real symmetric matrix can be diagonalized by an orthogonal transformation, so it suffices to consider diagonal B. By rescaling the axes we can reduce it to having entries $0, 1, -1$ on the diagonal. Now quadratic forms of such form are distinguished by the dimension of maximal subspace on which it is positive / identically zero.
	
	Similarly any complex quadratic form $Q$ can be represented by a Hermitian matrix $B$ such that 
	$$Q(x) = x^\dag B x$$
	By spectral theorem any hermitian matrix can be diagonalized by a unitary matrix, so it suffices to consider diagonal B.
	By rescaling the axes we can reduce it to having entries $0, 1$ on the diagonal. Now quadratic forms of such form are distinguished by the dimension of maximal subspace on which it is positive / identically zero.
\end{que}
\begin{que} What is $\text{gr}  \text{Cl}(Q)$
\end{que}
\begin{que} Classify $ \text{Cl}(Q)$ over $\mathbb{R}$ and $\mathbb{C}$ for $\dim V = 0, 1, 2$.

\end{que}
\begin{que} Find invertible elements in $\Lambda V$
	
	Clearly any element with zero scalar part is not invertible. Let $x$ be any such element. Suppose $x$ is a sum of $k$ decomposable vectors. Then $\underbrace{x \wedge ... \wedge x}_\text{k+1 times}$ consists of exterior products of $k+1$ decomposable vectors , hence contains a repetition, hence is zero. So $x$ is nilpotent. Thus $1+x$ is invertible. Thus, multiplying by a suitable scalar we get that any element with non-zero scalar part is invertible.
	
\end{que}
\end{document}
