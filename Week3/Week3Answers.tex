\documentclass[a4paper]{article}

\input{preamble.tex}
\usepackage{tikz-cd}
\usepackage[utf8]{inputenc}
\usepackage[T1]{fontenc}
\usepackage{textcomp}
\usepackage[english]{babel}
\usepackage{amsmath, amssymb}
\newtheorem{thm}{Theorem}
\newtheorem{lem}[thm]{Lemma}
\newtheorem{exmp}[thm]{Example}                                                                 
\newtheorem{defn}[thm]{Definition}
\newtheorem{que}[thm]{Question}   

\pdfsuppresswarningpagegroup=1

\begin{document}
\section*{ 2020-06-25}

\begin{que} Find a space $Y$ on which $F_2/F_\infty$ acts such that the quotient is wedge of two circles.

\end{que} 
\begin{que} For the deformed product to be associative, what properties must $m_1$ have?
	
	
\end{que}
\begin{que} Classify quadratic forms up to isomorphism when $K = \mathbb{C}$ or $K = \mathbb{R}$. 
	

\end{que}
\begin{que} What is $\text{gr}  \text{Cl}(Q)$
\end{que}
\begin{que} Classify $ \text{Cl}(Q)$ over $\mathbb{R}$ and $\mathbb{C}$ for $\dim V = 0, 1, 2$.\\
 \begin{enumerate}
     \item $K=\C$
         \begin{itemize}
             \item dim 0 : the tensor algebra is just scalars, the ideal we quotient by is trivial so the algebra is just $\C$
             \item dim 1 : If the quadratic form  has signature \{0\}, then we just get the exterior algebra. If \{1\}, then if $e_1$ is the basis vector of $V$ we get $e_1^2=1$ and that's the only condition on our algebra.
             \item dim 2 : For \{0,0\} it's just the exterior algebra. For \{1,0\} and a basis $e_1,e_2$ of $V$ we have by expanding  $1=\left( e_1+e_2 \right)^2=e_1e_1+e_2e_2+e_2e_1+e_1e_2=1=0+e_2e_1+e_1e_2 $ so $e_1e_2=-e_2e_1$. So our algebra is spanned by $1,e_1,e_2,e_1e_2$ modulo $e_1^2=1$,$e_2^2=0$,$e_1e_2+e_2e_1=0$. It can be represented as a matrix algebra with $e_1=\begin{pmatrix} 1 & 0\\0 & -1 \end{pmatrix} $, $e_2=\begin{pmatrix} 0 & 1\\0 & 0 \end{pmatrix} $.\\
                 Finally we have \{1,1\} which yields similar conditions except that now $e_2^2=1$. A possible matrix representation is one with $e_1$ as before and $e_2=\begin{pmatrix} 0 & i\\-i & 0  \end{pmatrix} $.
         \end{itemize}
 \end{enumerate}

\end{que}
\end{document}
