\documentclass[a4paper]{article}

\input{preamble.tex}
\usepackage{tikz-cd}
\usepackage[utf8]{inputenc}
\usepackage[T1]{fontenc}
\usepackage{textcomp}
\usepackage[english]{babel}
\usepackage{amsmath, amssymb}
\newtheorem{thm}{Theorem}
\newtheorem{lem}[thm]{Lemma}
\newtheorem{exmp}[thm]{Example}                                                                 
\newtheorem{defn}[thm]{Definition}
\newtheorem{que}[thm]{Question}   

\pdfsuppresswarningpagegroup=1

\begin{document}
\section*{ 2020-06-25}

\begin{que} Find a space $Y$ on which $F_2/F_\infty$ acts such that the quotient is wedge of two circles.

\end{que} 
\begin{que} For the deformed product to be associative, what properties must $m_1$ have?
	
	Keeping terms up to order $t$ we have:
	$$m(m(r_1, r_2), r_3) = m_0(m_0(r_1, r_2) + t m_1(r_1,r_2), r_3) + t m_1(m(r_1, r_2), r_3)) = $$
	$$ = m_0(m_0(r_1, r_2), r_3) + t (m_0(m_1(r_1, r_2), r_3)) +  m_1(m_0(r_1, r_2), r_3))$$
	And similarly for $m(r_1, m(r_2, r_3))$. Hence we need 
	$$m_0(m_1(r_1, r_2), r_3)) +  m_1(m_0(r_1, r_2), r_3) = m_0(r_1, m_1(r_2, r_3)) +  m_1(r_1, m_0(r_2, r_3))$$.
\end{que}
\begin{que} Classify quadratic forms up to isomorphism when $K = \mathbb{C}$ or $K = \mathbb{R}$. 
	
	
	Clearly any real quadratic form $Q$ can be (uniquely) represented by a symmetric matrix $B$ such that
	$$Q(x) = x^T B x$$
	Any real symmetric matrix can be diagonalized by an orthogonal transformation, so it suffices to consider diagonal B. By rescaling the axes we can reduce it to having entries $0, 1, -1$ on the diagonal. Now quadratic forms of such form are distinguished by the dimension of maximal subspace on which it is positive / identically zero.
	
	Similarly any complex quadratic form $Q$ can be represented by a Hermitian matrix $B$ such that 
	$$Q(x) = x^\dag B x$$
	By spectral theorem any hermitian matrix can be diagonalized by a unitary matrix, so it suffices to consider diagonal B.
	By rescaling the axes we can reduce it to having entries $0, 1$ on the diagonal. Now quadratic forms of such form are distinguished by the dimension of maximal subspace on which it is positive / identically zero.
\end{que}
\begin{que} What is $\text{gr}  \text{Cl}(Q)$
\end{que}
\begin{que} Classify $ \text{Cl}(Q)$ over $\mathbb{R}$ and $\mathbb{C}$ for $\dim V = 0, 1, 2$.\\
 \begin{enumerate}
     \item $K=\C$
         \begin{itemize}
             \item dim 0 : the tensor algebra is just scalars, the ideal we quotient by is trivial so the algebra is just $\C$
             \item dim 1 : If the quadratic form  has signature \{0\}, then we just get the exterior algebra. If \{1\}, then if $e_1$ is the basis vector of $V$ we get $e_1^2=1$ and that's the only condition on our algebra.
             \item dim 2 : For \{0,0\} it's just the exterior algebra. For \{1,0\} and a basis $e_1,e_2$ of $V$ we have by expanding  $1=\left( e_1+e_2 \right)^2=e_1e_1+e_2e_2+e_2e_1+e_1e_2=1=0+e_2e_1+e_1e_2 $ so $e_1e_2=-e_2e_1$. So our algebra is spanned by $1,e_1,e_2,e_1e_2$ modulo $e_1^2=1$,$e_2^2=0$,$e_1e_2+e_2e_1=0$. It can be represented as a matrix algebra with $e_1=\begin{pmatrix} 1 & 0\\0 & -1 \end{pmatrix} $, $e_2=\begin{pmatrix} 0 & 1\\0 & 0 \end{pmatrix} $.\\
                 Finally we have \{1,1\} which yields similar conditions except that now $e_2^2=1$. A possible matrix representation is one with $e_1$ as before and $e_2=\begin{pmatrix} 0 & i\\-i & 0  \end{pmatrix} $.
         \end{itemize}
     \item K=$\R$
         \begin{itemize}
             \item dim 0 : Same as $\C$ it's just $\R$
             \item dim 1 : \{0\} and \{1\} signatures are identical to the earlier case except that the scalars are different. \{-1\} gives $e_1^2 = -1$, so we get the algebra of complex numbers
             \item dim 2 : \{1, 1\} gives the same result as in previous case but with different scalars. In \{1, -1\} we have $e_1^2 = 1$ and $e_2^2 = -1$. Also $0 = (e_1 + e_2)^2 = e_1^2 + e_2^2 + e_1 e_2 + e_2 e_1$, so $e_1 e_2 = -e_1 e_2$. So our algebra is spanned by $e_1, e_2$ and $e_1e_2$ modulo $e_1^2 = 1$, $e_2^2 = -1$, $e_1 e_2 = -e_1 e_2$. I corresponds to a matrix algebra generated by $\begin{pmatrix} 0 & i\\-i & 0 \end{pmatrix} $ and $\begin{pmatrix} i & 0 \\ 0 & -i \end{pmatrix}$. In \{-1, -1\} we have $e_1^2 = e_2^2 = -1$ and $ e_1 e_2 = - e_1 e_2$ so we get the quaternions.
                 
                 
         \end{itemize}
 \end{enumerate}

\end{que}
\begin{que} Find invertible elements in $\Lambda V$
	
	Clearly any element with zero scalar part is not invertible. Let $x$ be any such element. Suppose $x$ is a sum of $k$ decomposable vectors. Then $\underbrace{x \wedge ... \wedge x}_\text{k+1 times}$ consists of exterior products of $k+1$ decomposable vectors , hence contains a repetition, hence is zero. So $x$ is nilpotent. Thus $1+x$ is invertible. Thus, multiplying by a suitable scalar we get that any element with non-zero scalar part is invertible.
	
\end{que}
\end{document}
