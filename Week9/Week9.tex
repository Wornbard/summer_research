\documentclass[a4paper]{article}

\input{preamble.tex}
\usepackage[utf8]{inputenc}
\usepackage[T1]{fontenc}
\usepackage{textcomp}
\usepackage[english]{babel}
\usepackage{amsmath, amssymb}
\newtheorem{thm}{Theorem}
\newtheorem{lem}[thm]{Lemma}
\newtheorem{exmp}[thm]{Example}
\newtheorem{defn}[thm]{Definition}
\newtheorem{que}[thm]{Question}
\newtheorem{clm}[thm]{Claim}
\pdfsuppresswarningpagegroup=1

\begin{document}
Suppose there is only one independent monomial $x^a y^b$. Then
$W(x, y) = P(x^a y^b)$ for some Laurent polynomial P.
Hence:
$$\nabla W(x,y) = (a x^{a-1} P'(x^a y^b), b x^{b-1} P'(x^a y^b))$$
At least one of $a, b$ is not zero, so at critical points we need $P'(x^a y^b) = 0$.
So, critical points have the form $$x^a y^b = c$$ with $c$ being one of the critical points of $P$.
Suppose that $P$ has a critical point $c$ not equal to $0$.
Let's look for elements of $\Gamma$ fixing critical points satisfying $x^a y^b =c$. Considering only first coordinate, the map has to look like:
$$(x,y) \mapsto x x^e y^f$$
With $x^e y^f=1$ whenever $x^a y^b = c$.
But we can arbitrarily multiply $x$ by a-th roots of unity and $y$ by b-th roots of unity, so it has to take form:
$$(x,y) \mapsto x x^{e a} y^{f b}$$
We can also multiply $x$ by $2^b$ and $y$ by $2^{-a}$, so we have to have $e=f$. Hence the only possible maps are:
$$(x,y) \mapsto x (x^{a} y^{b})^e$$
In fact, such map works, if $c$ is a root of unity, and $e$ divides order of $e$.
The most general form of the element of $\Gamma$ is therefore:
$$A = \begin{pmatrix}
	1+ k a & m a \\ 
	k b & 1+m b \\
\end{pmatrix}$$
We have $\det A = 1 + k a + m b$.
Suppose $A \in \Gamma_+$.

If $b \neq 0$, then $m = \frac{-k a}{b}$.
Then 
$$A = \begin{pmatrix}
1 + a k & \frac{- a^2 k}{b} \\ 
b k & 1 - a k \\
\end{pmatrix} $$
and 
$$A^n = \begin{pmatrix}
1 + n a k & \frac{- n a^2 k}{b} \\ 
n b k & 1 - n a k \\
\end{pmatrix} $$
Hence $\Gamma_+ \cong \mathbb{Z}$.
If $b = 0$ we can use the same argument with $a$.

In both cases we have $ \begin{bmatrix}
a\\ 
b 
\end{bmatrix}$ is an eigenvector of $A$ with eigenvalue $1$, so $A$ fixes $W(x,y)$.

Suppose $\det A  = -1$.

If $b \neq 0$, then $m = \frac{-2-k a}{b}$.
Then 
$$A = \begin{pmatrix}
1 + a k & \frac{a(-2-a k)}{b} \\ 
b k & -1 - a k \\
\end{pmatrix} $$
and 
$$A^2 = I $$

So $\Gamma$ is not isomorphic to $\mathbb{Z} \times C_2$, because it would contain elements of determinant $-1$ that square to non-identity elements.

The same argument also seems to work for higher dimensions.

\subsection*{Simplifying the conjugation condition}
We claim that given an element $A$ of $\Gamma$ s.t. $A^{T}QA=Q$, the condition is sufficient for each element of $A$ to give an inner (up to the sign change) autormorphism of $\text{Cl}\left( Q \right)$.\\
First note that WLOG (i.e. in an appropriate basis) $Q=\text{id}$. Then the condition says that in the corresponding basis $A\in O(n)$\\
In dimension 2 conjugating by $1+c e_1e_2$ gives $\frac{1}{1+c^2}$\begin{pmatrix}  $1-c^2$ & $-2c$\\$1+c^2$ & $1-c^2$ \end{pmatrix} i.e. an arbitrary element of $SO(2)$. Now for a general element of $O(2)$ we just need to get $-id$ via  $e_i\mapsto -1\cdot x^{-1}e_i x$ which is just conjugating by $1$. Hence we can get any element of $O(2)$ via conjugation (with the minus sign).

We know that every element of $SO(2)$ can yield an inner automorphism. Let's try to extend this by induction to arbitrary $n$.\\
Let's start with elements of  $SO(n)$. Then we want to get a proper inner automorphism. To proceed we need an useful fact: $SO(n)$ is generated by 2d rotations. That is because every element of $SO(n)$ has an invariant line or plane. Hence the vector space it acts on can be written as a direct sum of invariant lines/planes (by induction) and then we just need to rotate in each plane.\\
Now consider an arbitrary $A\in SO(n)$.
There is an orthogonal change of basis (i.e. one which preserves $Q=\text{id}$) s.t. the basis vectors of the Clifford algebra are the eigenvectors of $A$. Then we can realize $A$ on the invariant lines/planes by conjugation because in dimension 1 there is nothing to do and in dimension 2 we already know it's possible to conjugate by an element of even grading to get an automorphism induced by an arbitrary orthogonal matrix. Then conjugating by the product of elements giving these automorphisms on the subspaces, we get an automorphism of the whole algebra (because elements of even grading consisting of different basis vectors commute).\\
With $SO(n)$ out of the way let's try $O(n)\setminus SO(n)$. First note that $A\in O(n)\setminus SO(n)$ has an eigenvector with eigenvalue $-1$( $-\det(A+I)=\det(A^{T}\det(A+I)=\det(A+I)$ hence both sides 0). Like earlier pick a basis of eigenvectors s.t. $e_n$ corresponds to the eigenvector found above. Then we can realize the automorphism on the subspace without $e_n$ by conjugation, say by an element $x$ of even grading. Then conjugation by $e_n x$ gives $A$ with the opposite sign on all basis vectors which is what we want.

\end{document}
