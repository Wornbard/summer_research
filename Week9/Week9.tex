\documentclass[a4paper]{article}

\input{preamble.tex}
\usepackage[utf8]{inputenc}
\usepackage[T1]{fontenc}
\usepackage{textcomp}
\usepackage[english]{babel}
\usepackage{amsmath, amssymb}
\newtheorem{thm}{Theorem}
\newtheorem{lem}[thm]{Lemma}
\newtheorem{exmp}[thm]{Example}
\newtheorem{defn}[thm]{Definition}
\newtheorem{que}[thm]{Question}
\newtheorem{clm}[thm]{Claim}
\pdfsuppresswarningpagegroup=1

\begin{document}
Suppose there is only one independent monomial $x^a y^b$. Then
$W(x, y) = P(x^a y^b)$ for some Laurent polynomial P.
Hence:
$$\nabla W(x,y) = (a x^{a-1} P'(x^a y^b), b x^{b-1} P'(x^a y^b))$$
At least one of $a, b$ is not zero, so at critical points we need $P'(x^a y^b) = 0$.
So, critical points have the form $$x^a y^b = c$$ with $c$ being one of the critical points of $P$.
Suppose that $P$ has a critical point $c$ not equal to $0$.
Let's look for elements of $\Gamma$ fixing critical points satisfying $x^a y^b =c$. Considering only first coordinate, the map has to look like:
$$(x,y) \mapsto x x^e y^f$$
With $x^e y^f=1$ whenever $x^a y^b = c$.
But we can arbitrarily multiply $x$ by a-th roots of unity and $y$ by b-th roots of unity, so it has to take form:
$$(x,y) \mapsto x x^{e a} y^{f b}$$
We can also multiply $x$ by $2^b$ and $y$ by $2^{-a}$, so we have to have $e=f$. Hence the only possible maps are:
$$(x,y) \mapsto x (x^{a} y^{b})^e$$
In fact, such map works, if $c$ is a root of unity, and $e$ divides order of $e$.
The most general form of the element of $\Gamma$ is therefore:
$$A = \begin{pmatrix}
	1+ k a & m a \\ 
	k b & 1+m b \\
\end{pmatrix}$$
We have $\det A = 1 + k a + m b$.
Suppose $A \in \Gamma_+$.

If $b \neq 0$, then $m = \frac{-k a}{b}$.
Then 
$$A = \begin{pmatrix}
1 + a k & \frac{- a^2 k}{b} \\ 
b k & 1 - a k \\
\end{pmatrix} $$
and 
$$A^n = \begin{pmatrix}
1 + n a k & \frac{- n a^2 k}{b} \\ 
n b k & 1 - n a k \\
\end{pmatrix} $$
Hence $\Gamma_+ \cong \mathbb{Z}$.
If $b = 0$ we can use the same argument with $a$.

In both cases we have $ \begin{bmatrix}
a\\ 
b 
\end{bmatrix}$ is an eigenvector of $A$ with eigenvalue $1$, so $A$ fixes $W(x,y)$.

Suppose $\det A  = -1$.

If $b \neq 0$, then $m = \frac{-2-k a}{b}$.
Then 
$$A = \begin{pmatrix}
1 + a k & \frac{a(-2-a k)}{b} \\ 
b k & -1 - a k \\
\end{pmatrix} $$
and 
$$A^2 = I $$

So $\Gamma$ is not isomorphic to $\mathbb{Z} \times C_2$, because it would contain elements of determinant $-1$ that square to non-identity elements.

The same argument also seems to work for higher dimensions.
\end{document}
