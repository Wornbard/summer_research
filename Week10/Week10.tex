\documentclass[a4paper]{article}

\input{preamble.tex}
\usepackage[utf8]{inputenc}
\usepackage[T1]{fontenc}
\usepackage{textcomp}
\usepackage[english]{babel}
\usepackage{amsmath, amssymb}
\newtheorem{thm}{Theorem}
\newtheorem{lem}[thm]{Lemma}
\newtheorem{exmp}[thm]{Example}
\newtheorem{defn}[thm]{Definition}
\newtheorem{que}[thm]{Question}
\newtheorem{clm}[thm]{Claim}
\pdfsuppresswarningpagegroup=1

\begin{document}
\textbf{Idea 1}
Let $x = (x_1, x_2, ..., x_n)$ be in the domain of $W$.
Write $\textbf{x}^v$ for $x_1^{v_1} x_2^{v_2} .. x_n^{v_n}$. Then, if a matrix $M$ fixes $\textbf{x}$, then we have $\textbf{x}^v = \textbf{x}^{Mv}$ for any vector $v$.
Consider a term 
$$f(\textbf{x}) = \sum_{i=0}^{k-1} \textbf{x}^{M^i v}$$
where $M^k v =v$.
Then we have(assuming $M$ fixes $x$):
$$(\nabla f)_j = \sum_{i=0}^{k-1} (M^i v)_j \textbf{x}^{M^i v} = \sum_{i=0}^{k-1} (M^i v)_j \textbf{x}^{v} = (\sum_{i=0}^{k-1} M^i) \textbf{x}^{v} v_j$$
So $$\nabla f = (\sum_{i=0}^{k-1} M^i) \textbf{x}^{v} v$$.
$$(M-I)\nabla f= (M^k-I)v \textbf{x}^{v}= 0$$
Now $\Gamma$ fixes $W$, so we can partition $W$ into linear combination of terms like $f$, so:
$$(M-I)\nabla W = 0$$
For any $M$ in $\Gamma$.
Hence $\nabla W \in \cap_{M \in \Gamma} \ker(M - I)$

This shows that 
\begin{enumerate}
	\item If there is an element of $\Gamma$ with no eigenvalue $1$ then critical points of $W$ are precisely points fixed by it(additionaly it can be proven that in this case they are roots of unity of order divisible by $\det(M-I)$.(TODO: PROOF)
	\item As a consequence, all matrices in $\Gamma$ with no eigenvalue $1$ have to fix exactly the same points, and all other matrices also have to fix them.
	\item If all matrices in $M$ in $\Gamma$ have an eigenvalue $1$ but intersection of $ker(M-I)$ is empty, then $W$ has a critical point $(1, \dots,1)$ but also all other points fixed by all elements of $\Gamma$.
\end{enumerate}

Also this generalizes the idea with $-id$ in dimension 2. 

In fact, even if the kernels coincide, then except for some very rare cases $W$ has at least one critical point.

Summary of the idea:
How do points fixed by a matrix $M$ look like?

If $ker(M-I)$ is trivial, then all coordinates are roots of unity with order divisible by $det(M-I)$.

Otherwise, we can additionaly multiply by things of the form $(x^{v_1}, x^{v_2}, x^{v_3})$, where $(v_1, v_2, v_3) \in \ker(M-I)$, for example if $ M\begin{pmatrix}
1 & 0 &0 \\ 
0 & 0 & -1\\
0 & 1 & -1\\
\end{pmatrix}$ then fixed points are of the form $(x, \omega, \omega)$ where $\omega^3 = 1$.

Suppose $\dim (\ker(M-I))=1$. Then $\nabla W \in ker(M-I)$, and it's a polynomial in the free parameter describing the fixed points. Under some conditions(this is the messy part but it's definitely possible to formalize) we get a polynomial that is not of the form $a x^b$, and so has a non-zero complex root. So, then $W$ has a critical point.

This can also used to prove some restrictions on $W$ in certain cases(if different matrices in $\Gamma$ fix different classes of points then actually the polynomial above can't have any non-zero complex roots for the classes which are not fixed by all elements of $\Gamma$).

\textbf{Idea 2}
In dimension $2$:
\begin{enumerate}
	\item If $\Gamma_+ = C_2$, then fixed points are precisely $(1,1), (1,-1), (-1,1), (-1, -1)$. Additionally it can be proven that the quadratic form associated with $(1,1)$ is non-degenerate (using Cauchy Schwartz inequality on elements of a matrix one can prove that $\det Q > 0$).
	\item If $\Gamma = C_3$ or $\Gamma = S_3$, then $(1,1)$ is a critical point, critical points are exactly those fixed by an element of order $3$ (because $\det(M-I) = 3$), and they are third roots of unity.
	\item The above condition might imply that we can always find a transposition in case $C_3 < \Gamma$. We would then have that $C_3$ is not a possible $\Gamma$. TODO
	\item If $\Gamma = \mathbb{Z}$ or $\Gamma = D_\infty$ then:
	$W(x, y) = P(x^a y^b)$ and $P$ is a polynomial such that the only non-zero roots are roots of unity, and the non-zero roots are distinct(otherwise we would have critical points with zero quadratic form). Then all the critical points have a degenerate quadratic form with exactly one non-zero eigenvalue.
	If we knew that $P$ is a monic polynomial, then $P$ is a Kronecker polynomial(monic polynomial with roots being roots of unity) and there is a classification of them. Additionally imposing the condition that roots are distinct could give something more.
\end{enumerate}
\textbf{Idea 3}
In dimension 3:
If $W$ has a critical point with non-degenerate quadratic form, then we have to have $M^T Q M = Q$ for every $M \in \Gamma$, so $M$ is isomorphic to a subgroup of orthogonal group, which along with restricting order of elements to $1,2,3,4,6$ and the fact that if $-id \in \Gamma$ then it only has elements of order 2(as in dimension 2) shows that $\Gamma$ can be only one of $id, C_2, C_3, C_4, C_6, C_2 \times C_2, S_3, A_4, S_4, C_2 \times C_2 \times C_2, D_8$. In fact
\begin{enumerate}
	\item $id$ occurs for $W = 0$
	\item $C_2 \times C_2$ occurs for $CP^3$ with a 2d face blown up.
	\item $S_3$ occurs for $CP^3$ with a vertex blown up(PROOF TODO)
	\item $C_2 \times C_2 \times C_2$ occurs for cube.
	\item $S_4$ occurs for $CP^3$
	\item $D_8$ occurs for a square-based pyramid (PROOF TODO)
	\item $C_6$ occurs for $W(x,y,z)= 1/x y+y/x+1/(x z)+y/(x z)+ z/x+z/(x y)$
	\item We don't know yet for $C_2, C_3, C_4, A_4$, but as all of them are subgroups of some obtainable groups, it shouldn't be easy to exclude them.
\end{enumerate}
So if every element in $\Gamma$ has finite order, then either it is one of the above, or $\Gamma$ has no critical point with non-degenerate critical form.

Now suppose every critical point in $\Gamma$ has degenerate quadratic form.


\textbf{Case 1} There is a point with zero quadratic form.
Conjugation is trivial, so $\Gamma$ is trivial.

\textbf{Case 2} There is a point with only one non-zero eigenvalue.
The Clifford algebra is isomorphic to the one with $Q= \begin{pmatrix}
	1 & 0 &0 \\ 
	0 & 0 &0\\
	0 & 0 &0\\
\end{pmatrix}$.
Then inverse of $1+c e1e2+d e2e3+ e e1e3$ is $1-c e1e2-d e2e3- e e1e3$ and so conjugating by elements of even grading we can get automorphisms of the form(calculated with Mathematica):
$ \begin{pmatrix}
1 & 0 &0 \\ 
a & 1 & 0\\
b & 0 &  1\\
\end{pmatrix}$
Also, elements with zero scalar part aren't invertible so there is nothing to do.

Then inverse of $ce1+de2+ e e3+ e1e2e3$ is $1/c^2(ce1+de2+ e e3- e1e2e3)$ and so conjugating by elements of odd grading we can get automorphisms of the form(calculated with Mathematica):
$ \begin{pmatrix}
-1 & 0 &0 \\ 
a & 1 & 0\\
b & 0 &  1\\
\end{pmatrix}$
Elements with zero $e1e2e3$ are invertible, but they give the same kind of automorphisms.

That means that element of det $1$ give a subgroup of $\mathbb{Z} \times \mathbb{Z}$, and any element of determinant $-1$ is of order 2.
Hence $\Gamma$ can be isomorphic to either $\mathbb{Z} \times \mathbb{Z}$ or semidirect products: $\mathbb{Z} \times D_\infty$ or $D_\infty \times D_\infty$ or $C_2$.

\textbf{Case 3}. There is a point with two non-zero eigenvalues
Then inverse of $1+c e1e2+d e2e3+ e e1e3$ is $1/(1+c^2)(1-c e1e2-d e2e3- e e1e3)$ and so conjugating by elements of even grading we can get automorphisms of the form(calculated with Mathematica):
$\begin{pmatrix}
\frac{1-c^2}{1+c^2} & \frac{2c}{1+c^2} &0 \\ 
\frac{-2c}{1+c^2} & \frac{1-c^2}{1+c^2} & 0\\
\frac{2cd-2e}{1+c^2} & \frac{-2cd-2e}{1+c^2} &  1\\
\end{pmatrix}$
We can use a substitution(we can take $c=\infty$ to account for the fact that we set the scalar part to $1$) to get that the matrices are of the form:
$\begin{pmatrix}
\cos(\phi) & \sin(\phi) &0 \\ 
-\sin(\phi) & \cos(\phi) & 0\\
a & b &  1\\
\end{pmatrix}$

Inverse of $c e1+d e2+ e e3+ e1e2e3$ is $1/(c^2+d^2) (c e1 + d e2 + e e3 -1)$ and so conjugating by elements of odd grading we can get automorphisms of the form(calculated with Mathematica):
$\begin{pmatrix}
\frac{d^2-c^2}{c^2+d^2} & \frac{-2 d c}{c^2+d^2} &0 \\ 
\frac{-2 d c}{c^2+d^2} & \frac{d^2-c^2}{c^2+d^2} & 0\\
\frac{2d-2ce}{c^2+d^2} & \frac{-2c-2de}{c^2+d^2} &  1\\
\end{pmatrix}$
We can use a substitution to get that the matrices are of the form:
$\begin{pmatrix}
-\cos(\phi) & \sin(\phi) &0 \\ 
\sin(\phi) & -\cos(\phi) & 0\\
a & b &  1\\
\end{pmatrix}$
Then $\Gamma$ is isomorphic to $Z \times Z$ or $Z \times D_\infty$ or $D_\infty \times D_\infty$ or $C_2, C_3, C_4, C_6, D_4, S_3,D_8$.(PROOF TODO).

What we learned way too late:
Actually every finite subgroup of $GL(n, \mathbb{Z})$ is conjugate to a subgroup of $O(n)$ according to this:
https://mathoverflow.net/questions/27258/finite-subgroups-of-rm-sl-2-mathbbz-reference-request
So the classification from beginning of Idea 3 doesn't require considering different cases for $Q$.

\end{document}
