\documentclass[a4paper]{article}

% Some basic packages
\pdfminorversion=7

\usepackage[utf8]{inputenc}
\usepackage[T1]{fontenc}
\usepackage{textcomp}
\usepackage[english]{babel}
\usepackage{url}
\usepackage{graphicx}
\usepackage{float}
\usepackage{booktabs}
\usepackage{enumitem}

% Don't indent paragraphs, leave some space between them
\usepackage{parskip}

% Hide page number when page is empty
\usepackage{emptypage}
\usepackage{subcaption}
\usepackage{multicol}
\usepackage{xcolor}

% Other font I sometimes use.
% \usepackage{cmbright}

% Math stuff
\usepackage{amsmath, amsfonts, mathtools, amsthm, amssymb}
% Fancy script capitals
\usepackage{mathrsfs}
\usepackage{cancel}
% Bold math
\usepackage{bm}
% Some shortcuts
\newcommand\N{\ensuremath{\mathbb{N}}}
\newcommand\R{\ensuremath{\mathbb{R}}}
\newcommand\Z{\ensuremath{\mathbb{Z}}}
\renewcommand\O{\ensuremath{\emptyset}}
\newcommand\Q{\ensuremath{\mathbb{Q}}}
\newcommand\C{\ensuremath{\mathbb{C}}}
\newcommand\mathP{\ensuremath{\mathbb{P}}}
\newcommand{\tens}[1]{%
	\mathbin{\mathop{\otimes}\limits_{#1}}%
}
\DeclareMathOperator{\Aut}{Aut}
\DeclareMathOperator{\Inn}{Inn}
\DeclareMathOperator{\Hom}{Hom}
\DeclareMathOperator{\Tr}{Tr}
% Easily typeset systems of equations (French package)
\usepackage{systeme}

% Put x \to \infty below \lim
\let\svlim\lim\def\lim{\svlim\limits}

%Make implies and impliedby shorter
\let\implies\Rightarrow
\let\impliedby\Leftarrow
\let\iff\Leftrightarrow
\let\epsilon\varepsilon

% Add \contra symbol to denote contradiction
\usepackage{stmaryrd} % for \lightning
\newcommand\contra{\scalebox{1.5}{$\lightning$}}

% \let\phi\varphi

% Command for short corrections
% Usage: 1+1=\correct{3}{2}

\definecolor{correct}{HTML}{009900}
\newcommand\correct[2]{\ensuremath{\:}{\color{red}{#1}}\ensuremath{\to }{\color{correct}{#2}}\ensuremath{\:}}
\newcommand\green[1]{{\color{correct}{#1}}}

% horizontal rule
\newcommand\hr{
    \noindent\rule[0.5ex]{\linewidth}{0.5pt}
}

% hide parts
\newcommand\hide[1]{}

% si unitx
\usepackage{siunitx}
\sisetup{locale = FR}

% Environments
\makeatother
% For box around Definition, Theorem, \ldots
\usepackage{mdframed}
\mdfsetup{skipabove=1em,skipbelow=0em}
\theoremstyle{definition}
\newmdtheoremenv[nobreak=true]{definitie}{Definitie}
\newmdtheoremenv[nobreak=true]{eigenschap}{Eigenschap}
\newmdtheoremenv[nobreak=true]{gevolg}{Gevolg}
\newmdtheoremenv[nobreak=true]{lemma}{Lemma}
\newmdtheoremenv[nobreak=true]{propositie}{Propositie}
\newmdtheoremenv[nobreak=true]{definition}{Definition}
\newtheorem*{eg}{Example}
\newtheorem*{notation}{Notation}
\newtheorem*{previouslyseen}{As previously seen}
\newtheorem*{remark}{Remark}
\newtheorem*{note}{Note}
\newtheorem*{problem}{Problem}
\newtheorem*{observe}{Observe}
\newtheorem*{property}{Property}
\newtheorem*{intuition}{Intuition}
\newmdtheoremenv[nobreak=true]{prop}{Proposition}
\newmdtheoremenv[nobreak=true]{theorem}{Theorem}
\newmdtheoremenv[nobreak=true]{corollary}{Corollary}

% End example and intermezzo environments with a small diamond (just like proof
% environments end with a small square)
\usepackage{etoolbox}
\AtEndEnvironment{vb}{\null\hfill$\diamond$}%
\AtEndEnvironment{intermezzo}{\null\hfill$\diamond$}%
% \AtEndEnvironment{opmerking}{\null\hfill$\diamond$}%

% Fix some spacing
% http://tex.stackexchange.com/questions/22119/how-can-i-change-the-spacing-before-theorems-with-amsthm
\makeatletter
\def\thm@space@setup{%
  \thm@preskip=\parskip \thm@postskip=0pt
}


% \lecture starts a new lecture (les in dutch)
%
% Usage:
% \lecture{1}{di 12 feb 2019 16:00}{Inleiding}
%
% This adds a section heading with the number / title of the lecture and a
% margin paragraph with the date.

% I use \dateparts here to hide the year (2019). This way, I can easily parse
% the date of each lecture unambiguously while still having a human-friendly
% short format printed to the pdf.

\usepackage{xifthen}
\def\testdateparts#1{\dateparts#1\relax}
\def\dateparts#1 #2 #3 #4 #5\relax{
    \marginpar{\small\textsf{\mbox{#1 #2 #3 #5}}}
}

\def\@lecture{}%
\newcommand{\lecture}[3]{
    \ifthenelse{\isempty{#3}}{%
        \def\@lecture{Lecture #1}%
    }{%
        \def\@lecture{Lecture #1: #3}%
    }%
    \subsection*{\@lecture}
    \marginpar{\small\textsf{\mbox{#2}}}
}

\DeclareMathOperator{\Ima}{Im}

% These are the fancy headers
\usepackage{fancyhdr}
\pagestyle{fancy}

% LE: left even
% RO: right odd
% CE, CO: center even, center odd
% My name for when I print my lecture notes to use for an open book exam.
% \fancyhead[LE,RO]{Gilles Castel}

\fancyhead[RO,LE]{\@lecture} % Right odd,  Left even
\fancyhead[RE,LO]{}          % Right even, Left odd

\fancyfoot[RO,LE]{\thepage}  % Right odd,  Left even
\fancyfoot[RE,LO]{}          % Right even, Left odd
\fancyfoot[C]{\leftmark}     % Center

\makeatother




% Todonotes and inline notes in fancy boxes
\usepackage{todonotes}
\usepackage{tcolorbox}

% Make boxes breakable
\tcbuselibrary{breakable}

% Figure support as explained in my blog post.
\usepackage{import}
\usepackage{xifthen}
\usepackage{pdfpages}
\usepackage{transparent}
\newcommand{\incfig}[1]{%
    \def\svgwidth{\columnwidth}
    \import{./figures/}{#1.pdf_tex}
}

% Fix some stuff
% %http://tex.stackexchange.com/questions/76273/multiple-pdfs-with-page-group-included-in-a-single-page-warning
\pdfsuppresswarningpagegroup=1


% My name
\author{Jakub Wornbard}


\usepackage{tikz-cd}
\usepackage[utf8]{inputenc}
\usepackage[T1]{fontenc}
\usepackage{textcomp}
\usepackage[english]{babel}
\usepackage{amsmath, amssymb}
\newtheorem{thm}{Theorem}
\newtheorem{lem}[thm]{Lemma}
\newtheorem{exmp}[thm]{Example}                                                                 
\newtheorem{defn}[thm]{Definition}
\newtheorem{que}[thm]{Question}   

\pdfsuppresswarningpagegroup=1

\begin{document}
\section*{ 2020-06-18}

\begin{que} Can you embed the following groups into $F_2$?
	\begin{itemize}
		\item $F_3$ 
		
		Same as $F_\infty$, but restricted to first three generators.
		\item $F_\infty$ 
		
		Let $a, b$ be generators of $F_2$. Consider $x_i = b^{-i} a b^{-i}$ for $i = 1, 2, ...$.
			Let's prove that $\left< x_1, x_2, ... \right>$ is free:
			Suppose that some reduced word $w = x_{y_1}^{z_1} x_{y_2}^{z_2} ...$ is equal to identity.
			It is reduced, so $y_i \neq y_i+1$ for any $i$	
			$$b^{-y_1} a^{z_1} b^{y_1- y_2} a^{z_2} b^{y_2 - y_3} ... = e$$
			All exponents in the above word are non-zero, so the above word is reduced and non-empty, so it is not equal to identity, contradiction.
			Hence $\left< x_1, x_2, ... \right> \cong F_{\infty}$, so $F_2$ has a subgroup isomorphic to $F_{\infty}$.
			
	\end{itemize}
\end{que} 
\begin{que} How to recover $d_i$'s in classification of finitely generated abelian groups?
	
\end{que}
\begin{que} Show that free R-modules are projective.
	
	Let $p_1, ... p_k$ be a basis for a free R-module $P$. Suppose $f : N \to M$ is a surjective homomorphism and $g: N \to M$ is a homomorphism. $f$ is surjective, so let $n_i$ be s.t. $f(n_i) = g(p_i)$. 
	Now let $h : P \to M$ be defined by $h(r_i p_i)=r_i n_i$. This is well defined, because $P$ is free, and it's also a homeomorphism. Moreover $f \circ h(r_i p_i) = f(r_i n_i) = r_i g(p_i) = g(r_i p_i)$, so $g = f \circ h$, so $P$ is projective.
\end{que}
\begin{que} Show that if P is projective then $\otimes P$ preserves injectivity of maps.\\ \\
    Let's first prove the following: For any $R$-modules $M,N,Q$ we have  \[
        \left( M \oplus N \right)\otimes Q \cong \left( M\otimes Q \right) \oplus \left( N\otimes Q \right) 
    .\] 
    \begin{proof}
        Consider the bilinear map $f:\left( M\oplus N \right) \times Q\to \left( M\otimes Q \right) \oplus \left( N\otimes Q\right) $ given by $f\left( \left( m,n \right), q \right)=m\otimes q+n\otimes q $. Then by the fundamental property of tensor products the corresponding map $\phi:\left( M\oplus N \right) \otimes Q\to \left( M\otimes Q \right) \oplus \left( N\otimes Q  \right) $ that sends $\left( m,n \right) \otimes q$ to $m\otimes q+n\otimes q$ is a well-defined module homomorphism.\\
        We would now like to find its inverse. Consider the "obvious" maps $\varphi_N,\varphi_M:M\otimes Q,N\otimes Q\to \left( M\oplus N \right) \otimes Q$. Then the map $\psi:\left( M\otimes Q \right) \oplus \left( N\otimes Q \right)\to \left( M\oplus N \right) \otimes Q $ given by $\left( m\otimes q_m,n\otimes q_n \right)\mapsto \varphi_M(m\otimes q_m)+\varphi_N\left( n\otimes q_n \right)$ is a well-defined homomorphism since it's a sum of two homomorphisms. Now it remains to check that $\psi$ is an inverse of $\phi$ but this is a trivial check, so we are done
    \end{proof}
    Now let's show that taking a tensor product with a free module preserves the injectivity of maps. I will denote a free $R$-module by $R^{n}$ but everything that follows work just as well for modules of infinite rank. Let $f:M\to N$ be injective. Then we claim that $f\otimes id:M\otimes R^{k}\to N\otimes R^{k}$is injective. Suppose that $f\left( m\right)\otimes \left( r_1, \ldots, r_k \right)=f\left( m'\right)\otimes \left( r_1', \ldots,r_k' \right) $. We would like to show that this implies the equality of arguments. Since $M\otimes R^{k}\cong M^{k}$ we can instead assume $\left( r_1f(m), \ldots, r_kf(m) \right) = \left( r_1'f(m'), \ldots ,r_k'f(m') \right) $. But since $f$ is a homomorphism we can move the $r_i$and  $r_i'$ inside the argument and use the injectivity of $f$ to establish $r_im=r_i'm'$ which is exactly what we wanted, so we are done.\qed\\
    One more step in the proof (which, in case it isn't apparent, was created by reading a lot of simpler facts on wikipedia, trying to prove these first and seeing if these lead to anything) : For any projective module $P$ there exists a module $Q$ s.t. $P\oplus Q$ is free. Once again I will be assuming that $P$ has a generating set of size $n$ but $n=\infty$ works the same.\\
    Let  $P$ be generated by  $\left\{ p_1, \ldots, p_n \right\} $. Consider $f:R^{n}\to P$ that sends the i-th basis element to $p_i$. This is clearly a surjective module homomorphism. Then by the defining property of projective modules, there exists a homomorphism  $h:P\to R^{n}$ s.t. $f\circ h=id_P$
    \[
    \begin{tikzcd}
        R^{n}\arrow[rd,two heads,"f"]\\
        P\arrow[r,"id"]\arrow[u,dashed,hook,"h"] & P
    \end{tikzcd}
   \] 
   Now clearly $h$ is injective, because $f\circ h$ has trivial kernel, so  $\text{Im}(h) \cong P$. We claim that  $R^{n}\cong \text{im}\left( h \right) \oplus \text{ker}\left( f \right) $ and the isomorphism between these is \begin{align*}
       \left( r_1, \ldots, r_n \right)\mapsto \left( h\left( \sum r_i p_i \right),h\left( \sum r_i p_i \right)- \left( r_1, \ldots, r_n \right)  \right)=\\ =\left( h\circ f\left( r_i\right), h\circ f\left( r_i \right)-\left( r_i \right)    \right) 
   .
   \end{align*}
   \textbf{First question} : Why is the second term in the kernel of $f$? \\Because applying $f$ to both sides gives identity on first term and sth that cancels it on the second.\\
   \textbf{Bad question} : Is this a well-defined homomorphism? Yes.\\
   It is also a bijection, because $g : \text{Im}(h) \oplus \text{ker}(f) \to R^n$ given by $(a,b) \mapsto 
   a-b$ is a homomorphism and is inverse to this map.\qed \\
Finally, let's put it together. Given a projective module $P$ we can find $Q$ s.t. $P\oplus Q\cong R^{n}$. Then given an injective map $f:M\to N$ we get an injective map $M\otimes R^{n}\to N\otimes R^{n}$. Then using the distributivity of tensor product over direct sum (or the other way around) we see it's an injective homomorphism $\left( M\otimes P \right) \oplus \left( M\otimes Q \right) \to \left( N\otimes P \right) \oplus \left( N\otimes Q \right) $ and, importantly, the homomorphism had the $f\otimes id$ form on each of the summands. Now putting $0$ in the $\otimes Q$ entry this gives an injective homomorphism $M\otimes P\to N\otimes P$, which is what we wanted to show.
\end{que}
\begin{que} Show that $\Hom_{S-mod}(S \tens{R} M, N) = \Hom_{R-mod}(M, N_R)$.\\\\
 Consider a map $\varphi:\text{Hom}\left( S\otimes_R M,  N  \right) \to \text{Hom}\left( M,N_R \right) $ defined by \[\varphi\left( g \right) \left( m \right)=g\left( 1\otimes m \right)  \]
 It is clearly well defined. First let's note that it maps homomorphisms to homomorphisms since $1\otimes m+1\otimes n=1\otimes \left( m+n \right) $ and $\varphi(g)(rm)=g(1\otimes rm)=rg\left( 1\otimes m \right)=r\varphi(g)(m) $ where the r multiplication in the S-module $N$ is defined by some homomorphism $R\to S$ but it doesn't seem to matter how exactly it is defined. The map $\varphi$ is also a module homomorphism, since the sum of homomorphisms and a homomorphism multiplied by any $s$ is still a homomorphism. It remains to find an inverse to $\varphi$ in order to prove it's an isomorphism. Consider $\phi$ defined by $\phi(h)(s\otimes m)=sh(m)$. A similar check shows that it's well defined and a homomorphism. It also clearly is the inverse of $\varphi$ so the two homomorphism sets are isomorphic as $S$-modules.

\end{que}
\begin{que} Check why dividing by $\{m \otimes m : m \in M\}$ implies that $m_1 \otimes m_2 + m_2 \otimes m_1 = 0$ but the opposite doesn't hold in characteristic $2$\\\\
   \end{que}
\begin{que} Assume $M$ is free of rank $n$. What is $\Lambda^i M$?\\\\
    We will show that it is a free module of rank $\binom{n}{k}$. First note that $M^{\otimes k}\cong R^{n^{k}}$ and it is spanned by the tensor products of free basis elements of each copy of $M$. So the $k$-th exterior power, by definition, is spanned by the images of these. Now note that any basis tensor with two equal basis elements at two different positions is mapped to $0$ in the quotient (straightforward to check. We can use $a\otimes b=-b\otimes a$ to move two equal elements closer to each other until we get $m\otimes m=0$ ).\\
    Hence $\Lambda^{k}M$ is spanned by the $\binom{n}{k}$ basis elements not mapped to zero. So there is a surjective homomorphism $R^{\binom{n}{k}}\to \Lambda^{k}M$. We would like to show it's also injective. Let's assume a fact which we will show later, namely $\Lambda^{n}M$ is a free module of rank 1 spanned by $e_1\otimes \ldots\otimes e_n$. Now assume that the kernel of our map is non-trivial i.e. there exist $r_i$ s.t. $\sum r_i (e_{i_1}\otimes \ldots\otimes e_{i_k})=0$ in the exterior power. Then the image of the tensor product of this element with any tensor product of $n-k$ basis elements of $M$ is zero in the  $n$-th exterior power (by definition of an ideal). So we can pick products of basis elements such that after the tensor multiplication we are only left with  $\pm r_i \left( e_1 \otimes \ldots \otimes r_n \right) $. Then this has to be zero so $r_i$ is zero for all $i$ and so the kernel is trivial.\\
    It remains to show that the $\Lambda^{n}M$ is indeed a rank 1 free module. It is spanned by $e_1\otimes \ldots\otimes e_n$ (that is its image under the quotient map) by the earlier observation so just need to check that it is non-zero. \\Note that if an element is in the kernel of the quotient map $R^{n^{\otimes n}}\to \Lambda^{n}R^{n}$, then it is also in the kernel of any anti-symmetric map $R^{n^{\otimes n}}\to R$. Indeed elements in the kernel of the first map are in the ideal generated by elements of the form $m\otimes m$ (subset of it, containing products of $n$ terms). Then any element of such form is in the kernel of an anti-symmetric map (unless $R$ has characteristic 2. I'm not sure how it works then) because its image equals minus itself. So in the contrapositive form, if we can find an anti-symmetric map s.t. $e_1\otimes \ldots\otimes e_n$ is not in its kernel, then it is non-zero in the exterior power. An anti-symmetric map from the tensor product are the same as multilinear, anti-symmetric maps from the direct product but the latter are easier to define. So consider instead $f:R^{n}\times \ldots\times R^{n}\to R$ that sends $\left( r_{11},\ldots,r_{1n} \right),\ldots,\left(r_{n1},\ldots,r_{nn}\right) $ to $\sum_{\sigma\in S_n}\text{sgn}\left( \sigma \right) r_{i\sigma\left( i \right) }$. Then this clearly corresponds to a map that sends the product of basis vectors to $1\neq 0$ so $e_1\otimes \ldots\otimes e_n$ is indeed non-zero in the exterior power.
\end{que}
\begin{que} Given $f : M \to M$ a homomorphism of R-modules, write down two "interesting" maps
$$\Lambda^i M \to \Lambda^i M$$ induced by $f$. Interpret these maps when $i=n$.

Let $g_1 : \Lambda^i M \to \Lambda^i M$ be defined by $g_1(m_1 \otimes ... \otimes m_i) = f(m_1) \otimes ... f(m_i)$. 
Let $g_2 : \Lambda^i M \to \Lambda^i M$ be defined by $g_2(m_1 \otimes ... \otimes m_i) = f(m_1) \otimes ... \otimes m_i + m_1 \otimes f(m_2) ... \otimes m_i + ... + m_1 \otimes m_2 ... \otimes f(m_i)$. They are well defined because they factor through the quotient defining the exterior algebra.
If $i=n$, let $e_1, e_2, ..., e_n$ be the basis of $M$, and let $f_ij$ be the j-th component of $f(e_i)$ in this basis. The only basis vector for $\Lambda^n M$ is $e_1 \otimes e_2 ... \otimes e_n$:
$$g_1(C e_1 \otimes e_2 ... \otimes e_n) = C f(e_1) \otimes (e_2) ... \otimes f(e_n) = C (f_{1j}e_j) \otimes (f_{2j}e_j) ... \otimes (f_{nj}e_j) = C \det(f_{ij})  e_1 \otimes e_2 ... \otimes e_n$$
So $g_1$ corresponds to multiplication by $\det(f_{ij})$.
Similarly:
$$g_2(C( e_1 \otimes e_2 ... \otimes e_n)) = C(f(e_1) \otimes ... \otimes e_n + e_1 \otimes f(e_2) ... \otimes e_i + ... + e_1 \otimes e_2 ... \otimes f(e_n))$$
$$ = C(e_{11} + e_{22}+...+e_{nn}) e_1 \otimes e_2 ... \otimes e_n = C \Tr(f_{ij}) e_1 \otimes e_2 ... \otimes e_n$$
So $g_2$ corresponds to multiplication by $\Tr(f_{ij})$.
\end{que}

\end{document}
